\documentclass[12pt, openany, oneside, a4paper, english, brazil]{abntex2}   % Padrão Abntex2

%***************************** PACOTES *********************************** %
%\usepackage[within=none]{newfloat}
\usepackage[utf8]{inputenc}		     % Codificacao do documento (conversão automática dos acentos)


%	Fonte
\usepackage[T1]{fontenc}		   	 % Selecao de codigos de fonte.
\usepackage[brazil]{babel}
\usepackage{lmodern}			     % Usa a fonte Latin Modern

%	Matemático e Gráfico
\usepackage{float}
\usepackage{graphics,graphicx}	     % pacotes para inserir figuras .eps ou .jpg
\graphicspath{{figuras/}}            % pasta de figuras
\usepackage{amssymb}           		 % pacote para fontes e simbolos matemáticos
\usepackage{longtable}          	 % possibilita inserir grandes tabelas
\usepackage{makecell}                % possibilita pular linhas nas tabelas
\usepackage{xcolor,colortbl,multirow} % permite textos e tabelas com cores
\usepackage{amsmath}           		 % pacote para equações
\usepackage{epstopdf}
\usepackage{ragged2e}
\usepackage{listings}
%\lstset{numbers=left, numberstyle=\tiny, stepnumber=1, numbersep=5pt, basicstyle=\scriptsize ,frame=tbrl}
\usepackage{etoolbox}
\usepackage{multicol}
% \usepackage{caption}
% \usepackage{subcaption}
\usepackage[position=b,singlelinecheck=on]{subfig}

\floatstyle{plaintop}
\newfloat{quadro}{htbp}{loq}
\floatname{quadro}{Quadro}

\makeatletter
\patchcmd{\listof}% <cmd>
{\float@listhead}% <search>
{\@namedef{l@#1}{\l@table}\float@listhead}% <replace>
{}{}% <success><failure>
\makeatother

%\makeatletter
%\renewcommand*{\float@listhead}[1]{%
%	\@ifundefined{chapter}{%
%		\section*{#1}%
%		\addcontentsline{toc}{section}{#1}%
%	}{%
%	\chapter*{#1}% 
%	\addcontentsline{toc}{chapter}{#1}%
%}%
%\@mkboth{\MakeUppercase{#1}}{\MakeUppercase{#1}}%
%}
%\makeatother




%\DeclareFloatingEnvironment{quadro}[Quadro][Lista de quadros]
%\DeclareCaptionType{quadro}[Quadros][Lista de quadros]


% % % % % % % % % % % % % % % % % % % %

\definecolor{codegreen}{rgb}{0,0.6,0}
\definecolor{codegray}{rgb}{0.5,0.5,0.5}
\definecolor{codepurple}{rgb}{0.58,0,0.82}
\definecolor{backcolour}{rgb}{0.95,0.95,0.92}

%\definecolor{verde}{rgb}{0,0,0}

\lstdefinestyle{mystyle}{
	backgroundcolor=\color{backcolour},   
	commentstyle=\color{codegreen},
	keywordstyle=\color{magenta},
	numberstyle=\tiny\color{codegray},
	stringstyle=\color{codepurple},
	basicstyle=\footnotesize,
	breakatwhitespace=false,         
	breaklines=true,                 
	captionpos=b,                    
	keepspaces=true,                 
	numbers=left,                    
	numbersep=5pt,                  
	showspaces=false,                
	showstringspaces=false,
	showtabs=false,                  
	tabsize=2
}

\lstset{style=mystyle}

%	Estrutural
\usepackage{url}
\usepackage{threeparttable}     	 % permite a inserção de notas de rodapé nas tabelas
\usepackage{lscape}             	 % orientação de página LANDSCAPE
\usepackage{pifont}             	 % acrescenta símbolos diferentes.
%\usepackage{cmap}					 % Mapear caracteres especiais no PDF	
\usepackage{lastpage}				 % Usado pela Ficha catalográfica
\usepackage{indentfirst}			 % Indenta o primeiro parágrafo de cada sessão.



% Pacotes de citações
\usepackage[brazilian,hyperpageref]{backref}	 % Paginas com as citações na bibl
\usepackage[num]{abntex2cite}	% Citações padrão ABNT numerica
\citebrackets[]     
% remover bordas dos links
\hypersetup{
    colorlinks,
    linkcolor={black},
    citecolor={black},
    urlcolor={black}
}

\usepackage[round]{natbib}
\bibliographystyle{plainnat}
 % resto dos pacotes Latex
%\usepackage[english]{style/UFSC-ECA-Monograph} % personalização da ABNTEX2 para escrita em inglês
\usepackage[portuguese]{style/UFSC-ECA-Monograph} % personalização da ABNTEX2 para escrita em português





%---------------------------------------------------------------
%--------- DADOS BÁSICOS DO TRABALHO (Preencher todos) ---------
%---------------------------------------------------------------


\titulo{Bancada Experimental para Caracterização Aerodinâmica de VANTs em Ambiente Aberto com Escoamento Forçado}   % Título do Trabalho em Português
\tituloingles{Experimental Bench for Aerodynamic Characterization of UAVs on Open Environment with Forced Flow}   % Título do Trabalho em Inglês

\palavraschave{1. VANT. 2. Aerodinâmica. 3. Túnel de ento. 4. Sensoriamento remoto. 5. Aerodesign.}  % 5 palavras-chave
\keywords{1. UAV. 2. Aerodynamic. 3. Wind tunnel. 4. Remote sensoring. 5. Aerodesign.}   % 5 palavras-chave para resumo em inglês

\DEELautor{Rafael}{Araujo Lehmkuhl} % Aluno autor do trabalho

\curso{Engenharia Mecânica}												
\orientador{Prof. Dr. Amir Antônio Martins de Oliveira Júnior}  % Professor Orientador membro da banca 

\membrob{Prof. Dr. Carlos Henrique Nino Bohorquez}   % Professor membro da banca
\membroc{Prof. Dr. Edison da Rosa}   % Professor membro da banca

\local{Florianópolis}   % Cidade da instituição
\data{2019}        % Ano de realização


%--------------------------------------------------------------------------------------------



\begin{document}
% Seleção automática do idioma, não alterar
\ifx\isenglish\undefined
    \selectlanguage{brazil}
\else
    \selectlanguage{english}
\fi

% Texto da Dedicatória (OBS: dedicatória é opcional)
% Caso não queira inserir, deixe em branco ( \dedicatoria{} )
    
% Texto dos Agradecimentos (OBS: Agradecimentos é opcional)
% Caso não queira inserir, deixe em branco ( \agradecimentos{} )
\DEELagradecimentos{Agradeço a todos aqueles que, de alguma forma, auxiliaram para a concretização desta etapa, à equipe Céu Azul Aeronaves, responsável por manter acesa minha paixão pela engenharia, onde este trabalho nasceu e se desenvolveu, e em especial ao Valdecir e o Samuel. Sem eles este trabalho nunca teria acontecido.}

\DEELdedicatoria{Dedico este trabalho a meu pai e a minha mãe, que me proveram tudo o que eu sempre precisei para ser a melhor versão de mim. Aos meus avós e aos meus padrinhos, que me deram o suporte emocional e me guiaram a vida toda. A Mocha, que ignora tudo o que esta acontecendo pra me dar carinho e à Gabi, que me aguentou quase até o final da elaboração deste trabalho.\\}
                                  
% Texto da Epígrafe (OBS: Epígrafe é opcional)
% Caso não queira inserir, deixe em branco ( \epigrafe{} )
% \DEELepigrafe{\textit{"Tem que ver isso ai, talkei?."} \\ (BOLSONARO, Jair)} 
\DEELepigrafe{\textit{"Essa pica não é mais minha, essa pica agora é do aspira."} \\ (FÁBIO, Capitão)}
% \DEELepigrafe{\textit{"Bora bar."} \\ (MEMBRO DO AERO, Qualquer)} 
% \DEELepigrafe{\textit{"Pau no cu de quem não gostou."} \\ (LEHMKUHL, Rafael A.)} 
    

% Texto do Resumo (OBS: Resumo é Obrigatório)
\DEELresumo{No presente trabalho apresenta-se o desenvolvimento de uma bancada experimental, chamada de “túnel de vento externo”, para a medição do desempenho aerodinâmico de VANTs. Nesta bancada, a aeronave, ou componentes desta, são fixados em uma base instrumentada que é propulsionada horizontalmente em regime permanente por um veículo automotor, emulando as condições de velocidade e ângulo de ataque encontradas em voo. Os testes realizados nessa bancada têm por objetivo reduzir a necessidade de experimentação em túnel de vento nas etapas de desenvolvimento da aeronave e permitir a realização de testes rápidos e relativamente baratos da aeronave completa. A bancada é formada pela base, mecanismo de fixação da aeronave e instrumentação. A base que suporta a aeronave é projetada para ser fixada na carroceria de um veículo tipo camioneta. É formada por uma estrutura tubular treliçada com rigidez suficiente para evitar o movimento relativo da aeronave. A fixação da aeronave se dá por hastes perfuradas e pinos cuja movimentação permite o ajuste incremental do ângulo de ataque em valores pré-definidos. A base possui instrumentação para medição de sustentação, arrasto e momento, via células de carga, velocidade do escoamento e ângulo de ataque, via tubos de Pitot de múltiplas tomadas. Os testes são realizados em trechos retos de rodovia com pelo menos 3 km de extensão. Na velocidade de 20 m/s, essa distância permite 3 min de teste. O sensoriamento é automático e remoto via aplicativo no sistema operacional Android, para telefone celular. Os resultados obtidos para a aeronave da equipe Céu Azul para a competição de 2019 mostram que os valores de coeficiente de sustentação variam aleatoriamente em ± 13\% enquanto que o coeficiente de arrasto varia em ± 10\% durante movimento do veículo em velocidade constante. Os valores medidos do coeficiente de sustentação e coeficiente de arrasto em função do ângulo de ataque, no intervalo em 0 e 7 graus, são aproximadamente 60\% menores e 5\% menores, respectivamente, que os valores previstos em simulação usando o software AVL. Conclui-se que a bancada é operacional e capaz de estimativas comparativas do desempenho de diferentes soluções aerodinâmicas para um mesmo VANT. Recomenda-se para trabalhos futuros que seja aplicado um sistema que melhore a estabilidade do movimento do veículo em regime permanente através do controle automático de velocidade, a utilização de trechos retos com maior comprimento e pavimento plano, um estudo de transmissão de vibração do piso à base da bancada, uma melhoria na medição do arrasto para reduzir efeitos de atrito, e a realização de testes com formas simples e perfis de asa conhecidos para uma validação mais completa da precisão e exatidão das medições com a bancada.} 
     
% Texto do Abstract (OBS: Abstract é Obrigatório)
\DEELabstract{Here, the development of an experimental bench for the measurement of aerodynamic performance of UAVs is presented. In this bench, named an “outdoor wind tunnel”, the airplane, or some of its components, are fixed to an instrumented structure that is moved forward in steady state using a motored vehicle, emulating conditions found during flight. The tests in this bench have the objective to partially replace the need for wind tunnel tests during the development phase and to allow for fast and relatively cheap aerodynamic tests of the complete airplane. The bench is formed by a base, a fixture for the airplane and instruments. The base is designed to be bolted to the truck bed floor of a commercial pickup truck. It is made of a tubular frame with sufficient rigidity that prevents the relative motion of the airplane. The airplane is fixed using a perforated bean and pin that allows to change the angle of attack in prescribed steps. The base is instrumented with extensometers to measure lift, drag and momentum, and Pitot tubes to measure relative velocity and angle of attack. The tests are performed in straight horizontal tracks at least 3 km long, allowing for a test window of 3 min at 20 m/s. The measurements are remotely acquired using an application running in Android OP in a cell phone. The bench was used to measure the aerodynamic performance of the Céu Azul team airplane being prepared for the 2019 edition of the AeroDesign competition. During the test window, the values of lift and drag coefficients varied randomly in approximately ± 13\% and ± 10\% during the steady-state, constant speed movement of the vehicle. Also, the lift and drag coefficients measured for angles of attack from 0 to 7 degrees were consistently 60\% and 5\% smaller than the values predicted using the XFLR5 code. These differences are both ascribed to the experimental bench as well as to inaccuracies of the simulation code when applied to an entire airplane and additional tests are necessary to quantify the uncertainties in the measurements. It is concluded that the measurement bench is operational, the design decisions, except for the measurement of drag, are sound and the equipment can be used for fast and comparative evaluation of different aerodynamic solutions for the same airplane. It is recommended that a system that enforces constant speed in the vehicle be developed and used, as well as the use of longer, smoother tracks for future tests. The drag measurement must be improved and a vibration transmission study from the road to the airplane should also be performed. Finally, a thorough validation campaign using simple forms, such as spheres and cylinders, as well as, well known two-dimensional aerodynamic profiles, be developed in order to evaluate the uncertainties in the measurements at different Reynolds numbers and averaging elapsed time periods.} 
    



%%%%%%%%%%%%%%%%%%%%%%%%%%%
%% Elementos Pré-Textuais%%
%%%%%%%%%%%%%%%%%%%%%%%%%%%
\ElementosPreTextuais
% Insere todos os elementos Pré-Textuais listados (capa, folha de rosto, ...)
    % capa                   % Obrigatório
    % folhaderosto           % Obrigatório
    % folhaaprovacao         % Obrigatório
	% dedicatoria            % Opcional
	% agradecimentos         % Opcional
    % epigrafe               % Opcional
    % resumo                 % Obrigatório
    % abstract               % Obrigatório
\
%\newpage
%--------Lista de figuras--------
\ListaDeFiguras 
%--------Lista de tabelas--------
\ListaDeTabelas 
%--------Lista de Siglas e Abreviaturas--------
%----------Lista de Acronimos --------------------------------
\ifx\isenglish\undefined
\pretextualchapter{Lista de Siglas e Abreviaturas}
\else
\pretextualchapter{Acronyms}
\fi


%\renewcommand{\baselinestretch}{2} %espaçamento entre linhas (por definicçào, no espABNT é 2)
\begin{tabbing}
xxxxxxxxxxx \= xxxxxxxxxxxxx \kill
\textsc{ROS}            \> \textit{Robot Operating System}\\
\textsc{UFSC} \> \textit{Universidade Federal de Santa Catarina}\\
\end{tabbing}

%\addcontentsline{toc}{chapter}{Lista de Siglas e Abreviaturas}
   
%--------Lista de Simbolos e Notação--------
%\include{list_simbolos} 
%--------Sumário-------
\Sumario


%%%%%%%%%%%%%%%%%%%%%%
%%Elementos Textuais%%
%%%%%%%%%%%%%%%%%%%%%%
\textual 
\chapter{Introdução}\label{chp:intro}

Com a recente diminuição do custo das plataformas de processamento e sensoriamento embarcado [X] um novo ramo de mercado surgiu ao redor do desenvolvimento de VANTs [x].

Ainda que os avanços na tecnologia de baterias tenha permitido que este mercado surgisse, sua baixa densidade de potencia[X] exige que os projetos de VANTs sejam altamente otimizados em relação ao consumo energético.

Um problema frequente em projetos aeronáuticos, porém, é a incerteza quanto aos dados aerodinâmicos dos diversos componentes que compõem a aeronave. A resolução deste problema é abordada em geral de três maneiras: estimativas teóricas [X], simulação fluidodinâmica [X] e experimentação, possuindo cada um destes suas vantagens e desvantagens.

As estimativas teóricas em geral se encontram no projeto como método inicial de avaliação. A vantagem desta se dá pela vasta literatura disponível e por consistir na aplicação de fórmulas (em geral) simples. A desvantagem está na incerteza sobre a exatidão do resultado, isto é, não se sabe o quanto o mesmo esta próximo da realidade, ou mesmo se o valor é subestimado ou superestimado.

A simulação é uma alternativa muito utilizada, e sua vantagem se dá na maior exatidão dos resultados quando o modelo e os parâmetros de simulação já são bem conhecidos e comportados para a geometria estudada. O problema principal deste método se encontra na maior dificuldade de emprego, pois se limita a modelos que já tenham sido validados contra resultados experimentais. Para geometrias pouco estudadas, com parâmetros de simulação desconhecidos, corre-se o risco inclusive de se obter resultados ordens de grandeza afastados da realidade[X].

Por ultimo, a experimentação consiste em preparar um modelo físico semelhante ao projetado e ensaia-lo em um escoamento com as características desejadas. A dificuldade de emprego deste em geral é maior que a dos primeiros por exigir um trabalho de construção bem executado e equipamentos de medição com elevada precisão. Estas características fazem com que a experimentação tenha um custo elevado, muito diferente dos dois primeiros métodos, que tem custo praticamente nulo. Sua vantagem, porém, se dá no fato de que a certeza sobre os valores encontrados em geral é muito maior. Caso o teste tenha sido executado da maneira correta, os equipamentos de medição tenham precisão conhecida e sejam respeitadas as premissas do modelo, chega-se num resultado que pode ser tomado como real dentro de um intervalo de incerteza do equipamento de medição, isto é, diferente dos dois métodos anteriores, onde o resultado encontrado pode não corresponder com a realidade, neste existe uma faixa de certeza sobre o resultado.

\section{Sobre o projeto}

O principal problema do método experimental é a necessidade de equipamento especializado, e o método mais comum envolve a caracterização do modelo construído (em escala ou não) em um túnel de vento, com as forças medidas através de uma balança de precisão [X]. Túneis de vento, contudo, são equipamentos grandes e caros, e não disponíveis na maioria das universidades brasileiras. Túneis pequenos, como o da UFSC, são mais comuns porém não apresentam a possibilidade de medição de esforços aerodinâmicos em geometrias mais sofisticadas, devido a alta interferência no escoamento gerada pelas paredes do túnel. Componentes como asas, dispositivos de ponta de asa, conjuntos moto-propulsores, entre outros, que possuem grande relevância nos cálculos aerodinâmicos, acabam não podendo ter seu comportamento caracterizado nessas universidades.

\begin{figure}[!ht]
    \centering
    \includegraphics[width=.8\linewidth]{figuras/renders/perspectiva_sem_asa_com_pitot.png}
    \caption{Renderização da bancada modelada no software Solidworks 2016\cite{autor}.}
    \label{fig:render_bancada_completa}
\end{figure}

\begin{figure}[!ht]
    \centering
    \includegraphics[width=.8\linewidth]{figuras/construcao/bancada_inteira.jpg}
    \caption{Foto da bancada construída\cite{autor}.}
    \label{fig:bancada_construida}
\end{figure}

A proposta deste trabalho consiste no projeto e análise de uma bancada para medição de esforços aerodinâmicos sem túnel de vento, a ser utilizada para a caracterização aerodinâmica de componentes de VANTs. Tal bancada consiste de células de carga, tubos de pitot e outros sensores, e é embarcada em um veiculo automotor, que ao ser movimentado faz com que um escoamento surja sobre o componente a ser experimentado.

\section{Objetivo geral}

Desenvolvimento de uma bancada de baixo custo para caracterização aerodinâmica de componentes de VANTs, visando a eliminação ou minimização da necessidade de um túnel de vento.
 % Texto do cap1.tex
\chapter{Revisão de Literatura}\label{chp:rev}

\section{Aerodinâmica}

Quando um corpo se encontra imerso no escoamento de um fluido, forças e momentos surgem neste corpo resultantes da perturbação causada no fluido \citep{anderson1984fundamentals}. A força resultante pode ser decomposta em uma componente paralela a direção do escoamento, denominada arrasto, e outra perpendicular ao escoamento, denominada sustentação (ver figura \ref{fig:aero_profile_forces}). Escolhendo-se ainda um ponto arbitrário, surge um momento, que tende a girar o corpo ao redor desse ponto (ver figura \ref{fig:aero_profile_momentum}).

\begin{figure}[H]
    \centering
    \includegraphics[width=.8\linewidth]{figuras/outras/anderson_profile_forces.png}
    \caption{Forças de arrasto (D) e sustentação (L) geradas em um corpo sujeito a um escoamento com velocidade de fluxo livre $V_{\infty}$, sob um angulo de ataque $\alpha$. Fonte: \cite{anderson1984fundamentals}}
    \label{fig:aero_profile_forces}
\end{figure}

\begin{figure}[H]
    \centering
    \includegraphics[width=.8\linewidth]{figuras/outras/anderson_profile_momentum.png}
    \caption{Momento (M) gerado em um corpo imerso em um escoamento com velocidade $V_{\infty}$. Fonte: \cite{anderson1984fundamentals}}
    \label{fig:aero_profile_momentum}
\end{figure}

Tomando-se uma área e um comprimento de referência (ambos arbitrados) pode-se expressar estas forças e momentos através de coeficientes adimensionais  \citep{anderson1984fundamentals}. Para o caso do momento, o mesmo é medido usualmente (em superfícies sustentadoras) ao redor do ponto a um quarto de corda, devido ao fato de que, quando medido neste ponto, o coeficiente apresenta relativa invariância com o ângulo de ataque ($\alpha$) \citep{abbott1959theory}. 

\begin{equation}
    C_L =  \frac{L}{ \frac{1}{2} \rho V_{\infty}^{2} S}
\end{equation}

\begin{equation} 
    C_D =  \frac{D}{ \frac{1}{2} \rho V_{\infty}^{2} S}
\end{equation}

\begin{equation}
    C_M =  \frac{M}{ \frac{1}{2} \rho V_{\infty}^{2} S C_{M.A.}}
\end{equation}

onde $L$ representa a força de sustentação, $D$ a força de arrasto, $M$ o momento, $\rho$ a densidade do ar, $V_{\infty}$ a velocidade de fluxo livre, $S$ a área de referencia e $C_{M.A.}$ a corda média aerodinâmica da geometria de referência.

A vantagem em se utilizar estes coeficientes se da pela relativa invariância dos mesmos para pequenas faixas de número de Reynolds, o que permite que, uma vez conhecidos os coeficientes, possam ser calculadas as forças e momento aerodinâmicos apenas conhecendo-se as características do escoamento \citep{anderson1984fundamentals}.

\begin{figure}[H]
    \centering
    \includegraphics[width=.8\linewidth]{figuras/outras/drag_polar_prandtl.png}
    \caption{Exemplos de Polar de Arrasto para asas de diversas relações de envergadura sobre corda, em uma determinada faixa de Reynolds. O eixo das ordenadas representa $C_L$ e o eixo das abscissas $C_D$, ambos escalados por um fator de 100. Fonte: \cite{prandtl1921applications}}
    \label{fig:drag_polar_prandtl}
\end{figure}

É comum a apresentação gráfica dos coeficientes de sustentação em contraposição aos coeficientes em um mesmo gráfico (ver figura \ref{fig:drag_polar_prandtl}), para diferentes ângulos de ataque. Este gráfico é chamado Polar de Arrasto, e é uma representação gráfica do comportamento aerodinâmico de um corpo para uma determinada faixa de Reynolds \citep{prandtl1921applications}.

\section{Substituição do túnel de vento por bancada experimental}

Durante os estudos para desenvolvimento da aeronave X-57, no âmbito do projeto LEAPtech, \cite{murray2016leaptech} desenvolveram uma bancada de testes com funcionamento semelhante a da proposta neste trabalho. A bancada partiu da filosofia "all-up testing", proposta por George Mueller (NASA's Apollo) onde se reconhece a necessidade do teste integrado de sistemas (em oposição à realização somente de testes de sistemas isolados). Esta filosofia defende que em testes isolados o projetista procura por modos de falha conhecidos, enquanto em testes integrados o projetista busca a identificação de modos de falha desconhecidos, justamente pela dificuldade em se prever o comportamento dos sistemas quando não mais isolados. 

\begin{figure}[!ht]
    \centering
    \includegraphics[width=.8\linewidth]{figuras/leaptech/leaptech1.png}
    \caption{Bancada experimental desenvolvida no âmbito do projeto LEAPTech. Fonte: \cite{murray2016leaptech}}
    \label{fig:leaptech1}
\end{figure}

A proposta de construção desta bancada (frente ao uso de um túnel de vento) se deu devido à necessidade de se realizar iterações no projeto (e, consequentemente, testes) com maior agilidade, devido à natureza inovativa do projeto (propulsão elétrica distribuída ao longo da asa com beneficiamento por ganho local de velocidade).

\begin{figure}[H]
    \centering
    \includegraphics[width=.8\linewidth]{figuras/leaptech/cfd_comp.png}
    \caption{Comparações entre resultados de simulação CFD e dados experimentais no âmbito do projeto LEAPTech. Fonte: \cite{stoll2015comparison}}
    \label{fig:leaptech1}
\end{figure}

\cite{stoll2015comparison} realizou simulação CFD da asa da aeronave X-57 (com inclusão das hélices e seus efeitos) nos softwares STAR-CCM+ (RANS), FUN3D (RANS) e VSPAERO (VLM), e comparou os mesmos com os resultados obtidos na bancada desenvolvida. Foram encontradas diferenças menores que 10\% entre os resultados de simulação e os experimentais, indicando uma grande possibilidade deste método experimental ser valido para as avaliações desejadas.

\section{Considerações para o projeto da bancada}

\cite{gonzalez2011components} definem como principais pontos de atenção no projeto de uma balança para medição de forças aerodinâmicas os seguintes:

\begin{itemize}
    \item Calibração das células de carga individualmente - De acordo com os autores é necessário conhecer as incertezas de cada sensor individualmente para poder-se avaliar corretamente os efeitos de acoplamento. 
    \item Calibração da balança integrada - Mesmo com o conhecimento da incerteza das células individualmente, efeitos de acoplamento ocorrem nas medições, como resultado principalmente das deformações elásticas que ocorrem na estrutura da balança.
    \item Calibração para os efeitos dinâmicos - \cite{gonzalez2011components} ilustra valores típicos de carga na balança e vibrações induzidas pelo escoamento, mostrando que a ordem de grandeza da amplitude das cargas devido às vibrações pode ser a mesma da carga, comprometendo as medições.
    \item Avaliação da taxa de aquisição miníma dos sensores - No mesmo artigo é ilustrada a avaliação da taxa de amostragem dos dados de força através da sua média, mostrando que a média sofre uma oscilação para baixas taxas de aquisição até se estabilizar a partir de uma certa taxa de amostragem. Este mecanismo de avaliação pode ser utilizado em dados já adquiridos através da sub-amostragem dos dados.
    \item Filtragem dos dados - Segundo os autores, dado que as cargas devem ser invariantes no tempo para a situação estática, toma-se como aceitável a filtragem dos dados através de um filtro de médias-moveis.
\end{itemize}

\begin{figure}[!ht]
    \centering
    \caption{Exemplos de analises levantadas para garantir a qualidade dos dados medidos. Fonte: \cite{gonzalez2011components}}
        \subfloat[Demonstração de convergência do valor médio com o aumento da frequência de aquisição do dado.]{\includegraphics[width=0.4\columnwidth]{figuras/outras/convergence_rate.png}
        \label{convergence_rate}}
        \qquad
        \subfloat[Curva mostrando o acoplamento (indesejado) entre as forças de sustentação (L) e de arrasto (D), isto é, a falsa leitura de arrasto pelo sistema quando apenas a força de sustentação está presente.]{\includegraphics[width=0.4\columnwidth]{figuras/outras/coupled_forces.png}}
        \label{coupled_forces}
\end{figure}

As calibrações e análises estipuladas nos itens anteriores serão desenvolvidas ao longo deste trabalho.

% \section{Considerações quanto ao processamento dos dados gerados experimentalmente}
 % Texto do cap2.tex
\chapter{Metodologia}\label{chp:met}

\section{Projeto}

\subsection{Necessidades}

\subsubsection{O cliente}
O cliente do presente projeto é a equipe Céu Azul Aeronaves, que representa anualmente a UFSC na competição SAE Brasil Aerodesign. A equipe desenvolve VANTs com missões variadas, mas que (em geral) giram em torno da maximização da eficiência estrutural da aeronave projetada[X]. O autor do presente texto trabalhou durante 5 anos na equipe e viu nesta necessidade a oportunidade de desenvolvimento do presente trabalho.

O objetivo da bancada é medir os coeficientes aerodinâmicos (de sustentação, arrasto e momento) em componentes dos VANTs desenvolvidos na equipe, além da medição de empuxo dinâmico em conjuntos moto-propulsores e a interação de seu escoamento com o restante da aeronave.

Para cumprir com o objetivo proposto, algumas necessidades foram levantadas:

\begin{enumerate}
    \item Tal bancada deve possuir rigidez suficiente para que suas medições sejam consistentes no tempo e a necessidade de manutenção seja baixa.
    \item Deve ser possível acoplar a bancada na caçamba de uma camionete ou ao rack de um carro comum, de modo a facilitar sua instalação.
    \item Levando em conta que a equipe sofre de alta rotatividade de membros e nem sempre é possível garantir que existirão pessoas capacitadas a trabalhar neste projeto é importante que a bancada em questão seja entregue na forma de um "produto final", isto é, funcione sem a necessidade de conhecimento de suas intrinsidade, com pouco mais que um manual de instruções e considerações para seu uso.
    \item Da mesma forma que o uso da bancada deve ser facilitado, também deve ser assim o uso dos dados fornecidos por ela. Idealmente se deseja que a mesma entregue os coeficientes finais, já processados, assim como suas incertezas de medição.
\end{enumerate}

\subsection{Considerações}

Assume-se para o presente projeto que o escoamento de ar que alcança a geometria analisada tem direção relativamente estável, é aproximadamente paralela ao deslocamento do veiculo e que a velocidade é praticamente constante para pequenos trechos a serem analisados.

Dado que a velocidade do veiculo é controlada por um piloto humano, oscilações da ordem de 10\% na velocidade são esperadas durante os patamares de velocidade e serão mensuradas através de tubos de Pitot.

\begin{figure}[!ht]
    \centering
    \includegraphics[width=.8\linewidth]{figuras/renders/bancada_no_carro_lateral.png}
    \caption{Angulo de escoamento induzido pela carenagem do carro sobre a região de medições\cite{autor}.}
    \label{fig:vehicle-angle}
\end{figure}

É possível também que a carroceria do veiculo induza um angulo diferente de zero ao escoamento que chega a bancada (ver \prettyref{fig:vehicle-angle}). Um Pitot de múltiplas tomadas sera instalado na bancada a fim de medir este angulo e ajustar os dados no pós-processamento. Eh importante ressaltar que tal sonda  possui limitação de operação de aproximadamente 15 graus em cada sentido[X], sendo assim, ângulos maiores que este não terão uma leitura adequada. Grandes desvios de angulo do escoamento ou velocidades muito instáveis provavelmente tornarão o processamento dos dados dos testes impeditivo e portanto cuidados devem ser tomados para que o teste se conduza de forma adequada.

\begin{figure}[!ht]
    \centering
    \includegraphics[width=.8\linewidth]{figuras/renders/bancada_no_carro_curva_frontal.png}
    \caption{Surgimento de força radial sob o componente quando em curva\cite{autor}.}
    \label{fig:forca_radial_frontal}
\end{figure}

\begin{figure}[!ht]
    \centering
    \includegraphics[width=.8\linewidth]{figuras/renders/carro_momento_roll_frontal.png}
    \caption{Momento de rolagem artificial medido quando em curva\cite{autor}.}
    \label{fig:momento_falso_roll}
\end{figure}

\begin{figure}[!ht]
    \centering
    \includegraphics[width=.8\linewidth]{figuras/renders/bancada_no_carro_lateral.png}
    \caption{Oscilaçoes em pista causam erro na mediçao das forças e desalinhamento do escoamento\cite{autor}.}
    \label{fig:ocsilacoes_em_pista}
\end{figure}

Assume-se também que existem obstáculos na pista (tais como buracos e pequenas elevações), mas que de forma geral o teste sera conduzido em pistas relativamente retas e planas, de modo que não existam vibrações constantes ou acelerações radiais a serem modeladas no sistema. Pequenas oscilações serão medidas por acelerômetros e giroscópios e levadas em conta também no pós-processamento.

\subsection{Estimativas}

Para os projetos mecânico e eletrônico foram estimadas situações extremas de medição e para estas a bancada foi dimensionada:

\begin{itemize}
    \item Caracterização de aeronave completa com até 300N de sustentação, 50N de arrasto e XXXN de Momento
    \item Caracterização de conjunto motopropulsor com até 70N de empuxo estático
    \item Escoamentos com velocidade de até 25m/s
    \item Fatores de carga na bancada de até 4g
\end{itemize}

\subsection{Solução proposta}

A solução proposta consiste em uma bancada sensoriada, montada sobre uma estrutura metálica que a eleve até a altura desejada para o teste.

O veiculo a ser utilizado nos testes é do modelo Volkswagen Amarok 2017, tendo a caçamba uma altura de XXX m e o teto do carro elevando-se XXX m acima da altura da caçamba.

\begin{figure}[!ht]
    \centering
    \includegraphics[width=.8\linewidth]{figuras/renders/bancada_no_carro.png}
    \caption{ESQUEMATICO DA BANCADA + TORRE\cite{autor}.}
    \label{fig:placeholder}
\end{figure}

\subsubsection{Sensoriamento das forças}

As forças na bancada são sensoriadas por quatro células de cargas verticais e uma célula de carga horizontal. A célula horizontal permite medição do arrasto/empuxo, o somatório das células verticais permitem a medição da sustentação, a diferença entre as células frontais e traseiras permite a medição do momento de picada (pitch) e a diferença entre as células da esquerda e da direita permite a medição do momento de rolagem (roll).

\begin{figure}[!ht]
    \centering
    \includegraphics[width=.8\linewidth]{figuras/renders/esquematico_celulas_lift.png}
    \caption{Leitura de sustentação\cite{autor}.}
    \label{fig:leitura_sustentacao}
\end{figure}

\begin{figure}[!ht]
    \centering
    \includegraphics[width=.8\linewidth]{figuras/renders/esquematico_celulas_drag_2.png}
    \caption{Leitura de arrasto\cite{autor}.}
    \label{fig:leitura_arrasto}
\end{figure}

\begin{figure}[!ht]
    \centering
    \includegraphics[width=.8\linewidth]{figuras/renders/esquematico_celulas_momento_pitch.png}
    \caption{Leitura de momento de picada\cite{autor}.}
    \label{fig:leitura_momento_pitch}
\end{figure}

\begin{figure}[!ht]
    \centering
    \includegraphics[width=.8\linewidth]{figuras/renders/esquematico_celulas_momento_roll.png}
    \caption{Leitura de momento de rolagem\cite{autor}.}
    \label{fig:leitura_momento_roll}
\end{figure}

\subsubsection{Sensoriamento de velocidade e angulo de ataque}

Ainda na bancada serão instalados um tubo de Pitot e um Pitot de múltiplas tomadas (sonda de angulo de ataque). Ambos serão instalados o mais próximo possível ao escoamento livre. Enquanto o Pitot sera acoplado a própria bancada, a sonda ficara alinhada com o eixo principal de picada do componente que estiver sendo analisado, de modo a medir o angulo de ataque real do mesmo. 

\begin{figure}[!ht]
    \centering
    \includegraphics[width=.8\linewidth]{figuras/renders/detalhe_pitot_bancada.png}
    \caption{Posicionamento do tubo de Pitot na bancada para medição da velocidade de fluxo livre \cite{autor}.}
    \label{fig:placeholder}
\end{figure}

\begin{figure}[!ht]
    \centering
    \includegraphics[width=.8\linewidth]{figuras/placeholder.png}
    \caption{IMAGEM DA SONDA NA ASA \cite{autor}.}
    \label{fig:placeholder}
\end{figure}

O sistema deve comportar ainda a adição de novos sensores de forma a permitir a caracterização do escoamento em outros pontos de interesse, como esteira da hélice ou escoamento incidente no profundor.

\begin{figure}[!ht]
    \centering
    \includegraphics[width=.8\linewidth]{figuras/placeholder.png}
    \caption{IMAGEM DE PITOT NO PROFUNDOR \cite{autor}.}
    \label{fig:placeholder}
\end{figure}

\subsubsection{Sensoriamentos adicionais}

Para a medição do fator de carga vertical, assim como da temperatura e pressão do ar, uma Unidade de Medição Inercial (IMU) com acelerômetro, giroscópio, barômetro e termômetro será instalada na bancada.

\begin{figure}[!ht]
    \centering
    \includegraphics[width=.8\linewidth]{figuras/placeholder.png}
    \caption{FOTO DA IMU\cite{autor}.}
    \label{fig:placeholder}
\end{figure}

\subsubsection{Estação de controle e software de processamento}

Num primeiro momento foi levantada a possibilidade de se utilizar um computador portátil como estação de controle da bancada. Avaliou-se porém que isto poderia tornar o uso dificultado, já que em geral estes computadores possuem baterias que permitem poucas horas de uso continuo, limitando o tempo de execução de teste, além de serem relativamente grandes e pouco práticos de se utilizar dentro do carro.

De forma a facilitar o uso da bancada todo o controle do sistema foi projetado para ser realizado através de um aplicativo para celular. Nele é possível controlar a execução do teste, inserir informações para posterior avaliação, assim como acompanhar as medições dos sensores em tempo real, de modo a identificar possíveis problemas de forma rápida.

\begin{figure}[!ht]
    \centering
    \includegraphics[width=.8\linewidth]{figuras/placeholder.png}
    \caption{FOTO DO APLICATIVO\cite{autor}.}
    \label{fig:placeholder}
\end{figure}

Será desenvolvido ainda um software de tratamento e analise de dados com interface gráfica e uso facilitado, este a ser utilizado em computador, devido a maior flexibilidade do mesmo para trabalhos longos.

\section{Projeto Mecânico}

Para o projeto mecânico foram levantadas as seguintes necessidades:

\begin{itemize}
    \item Facilidade construtiva, de forma a tornar rápida a construção e manutenção da bancada
    \item Baixo peso, de modo a não se criar uma barreira quanto ao uso da mesma
    \item Baixo arrasto aerodinâmico, a fim de diminuir a influência da estrutura nas medições
    \item Rigidez, para que as forças e momentos medidos sejam correspondentes aos modelados
    \item Baixo custo
\end{itemize}

Uma solução que responde de forma positiva a maioria dessas necessidades é a de uma estrutura composta de perfis extrudados de alumínio. Este tipo de estrutura é comumente encontrada em laboratórios ou fabricas, sendo utilizada para a construção de bancadas experimentais e estações de trabalho.

\begin{figure}[!ht]
    \centering
    \includegraphics[width=.8\linewidth]{figuras/placeholder.png}
    \caption{FOTO DO PERFIL DE ALUMINIO\cite{autor}.}
    \label{fig:placeholder}
\end{figure}

A desvantagem desta estrutura fica no provável alto arrasto aerodinâmico. Este porém é um problema contornável posteriormente carenando-se a estrutura. Ainda assim, com ou sem carenagem, este arrasto deve ser levado em conta nos testes e a maneira levantada de se mensurar esta grandeza é realizar os testes sem um componente a ser medido, medindo-se assim o arrasto aerodinâmico da própria bancada para que se possa descontar este valor das medições posteriores.

De modo a se medir o arrasto, duas soluções foram consideradas: 

\begin{enumerate}
    \item Uma célula de carga horizontal, tendo a bancada liberdade de movimento no eixo X
    \item Três células de carga restringindo completamente os graus de liberdade da bancada
\end{enumerate}

\begin{figure}[!ht]
    \centering
    \includegraphics[width=.8\linewidth]{figuras/placeholder.png}
    \caption{ESQUEMATICO DAS SOLUCOES 1 E 2\cite{autor}.}
    \label{fig:placeholder}
\end{figure}

Devido ao fato de que a segunda solução exigiria ao menos três células de carga e as colocaria como componentes estruturais (já que não existiria outra ligação estrutural entre a parte inferior e superior da bancada) a primeira opção foi escolhida.

Para dar liberdade de movimentação no eixo X um conjunto de guias lineares e fixações rolamentadas foi escolhido, devido a seu baixo atrito e baixo custo. A desvantagem desta solução com apenas uma célula de carga se da justamente devido a existência desta força de atrito das guias no eixo X, que a principio não é medida. Esta força pode porem ser avaliada num primeiro momento, em laboratório, e compensada posteriormente.
    
\section{Projeto Eletrônico}

Para o sistema eletrônico da bancada tomou-se como ponto de partida a telemetria T2016, desenvolvida pela equipe Céu Azul inicialmente para utilização nos projetos da classe Advanced.

TROCAR TEXTO ABAIXO POR TABELA

Tal sistema consiste de um computador central, modelo Raspberry Pi 3 B+, rodando um sistema operacional Linux, com sensores conectados como periféricos. Entre os sensores já disponíveis na equipe estavam:

\begin{itemize}
    \item Tubos de Pitot com transdutores MPVX7002DP
    \item Modulo GNSS modelo uBlox NEO-6M
    \item IMU modelo GY85, com acelerômetro de 3 eixos, giroscópio de 3 eixos, magnetômetro de 3 eixos, barômetro e termômetro
\end{itemize}

Além destes sensores o sistema possui um par de rádios seriais de 433MHz, que pode ser utilizado para transmissão dos dados em tempo real para o celular, comunicando a bancada com o aplicativo controlado pelo operador.

Para a medição das forças na bancada foi necessária a aquisição de células de carga e transdutores de célula de carga. Devido ao baixo custo optou-se por células sem marca. Esta decisão implica contudo num custo extra de tempo para a caracterização das células.

Para a transdução dos dados da célula (de resistência para tensão e posteriormente para força) foram adquiridos módulos HX711. Estes módulos são bastante comuns no mercado e implementam num mesmo CI a alimentação e leitura da ponte de Wheatstone, além da conversão dos dados analógicos em dados digitais. A maior vantagem deste CI é provavelmente a alimentação integrada da ponte de Wheatstone, por ser previamente estabilizada e filtrada, resultando em uma alta relação sinal-ruido. Em geral essa alimentação acontece em circuitos discretos separados do CI de leitura, e acabam resultando numa pior relação sinal-ruido.

\section{Projeto de Software Embarcado}

O software que roda de forma embarcada na plataforma central consiste em uma serie de módulos escritos em Python com funções desmembradas. Entre as funções necessárias no software destacam-se:

\begin{itemize}
    \item Aquisição de dados dos sensores
    \item Parseamento dos dados para padrão comum
    \item Transmissão serial de dados via radio
    \item Recebimento e interpretação de comandos externos
    \item Gravação dos dados no sistema
    \item Coordenação e sincronia dos processos anteriores
\end{itemize}

A figura X mostra a arquitetura do software divida em seus diversos módulos.

\begin{figure}[!ht]
    \centering
    \includegraphics[width=.8\linewidth]{figuras/placeholder.png}
    \caption{ESQUEMATICO DO SOFTWARE EMBARCADO\cite{autor}.}
    \label{fig:placeholder}
\end{figure}

Este software já existia em versão primaria na plataforma T2016 e foi refatorado para uso na nova plataforma, permitindo que suas funções fossem estendidas. A totalidade das mudanças compreendidas por este trabalho pode ser vista no repositório Git do projeto [X]. 
\section{Projeto do Software de Analise}

O software de analise tem por função receber os dados "crus" gerados pela bancada e entregar dados uteis processados, com suas devidas incertezas estimadas.

Entre as características desejadas neste software estão:

\begin{itemize}
    \item Apresentar uma interface amigável para uso facilitado
    \item Receber os dados "crus"
    \item Filtrar cada dado conforme especificações previas ou personalizadas pelo usuário
    \item Apresentar os dados crus e processados na forma de gráficos
    \item Permitir interação do usuário com os dados
    \item Exportar os dados em formato útil, seja na forma textual, em planilhas ou mesmo diretamente como gráficos
\end{itemize}

A figura X mostra a arquitetura deste software.

\begin{figure}[!ht]
    \centering
    \includegraphics[width=.8\linewidth]{figuras/placeholder.png}
    \caption{ESQUEMATICO DO SOFTWARE DE ANALISE\cite{autor}.}
    \label{fig:placeholder}
\end{figure}

\section{MVP}

Para validar a ideia desta bancada foi proposto um MVP que possuísse as principais características da solução proposta, mas que pudesse ser construído com materiais já disponíveis pela equipe.

Do ponto de vista de projeto mecânico, esta primeira versão da bancada teve sua estrutura construída em madeira, utilizou corrediças de gaveta para dar liberdade de movimento no eixo X e usava como torre uma estrutura de aço. A escolha por estas soluções se deu pela já disponibilidade das mesmas na equipe Céu Azul.

Do ponto de vista de projeto eletrônico todos os componentes já estavam presentes, com exceção do Pitot de múltiplas tomadas, das células de carga e dos módulos HX711.

Do ponto de vista de software o aplicativo para controle da bancada foi desenvolvido de forma preliminar enquanto o software de tratamento e analise de dados ainda não existia.

As fotos a seguir mostram o MVP finalizado.

\begin{figure}[!ht]
    \centering
    \includegraphics[width=.8\linewidth]{figuras/outras/placeholder.png}
    \caption{FOTO DA BANCADA 1.0 - 1\cite{autor}.}
    \label{fig:placeholder}
\end{figure}

\begin{figure}[!ht]
    \centering
    \includegraphics[width=.8\linewidth]{figuras/outras/placeholder.png}
    \caption{FOTO DA BANCADA 1.0 - 2\cite{autor}.}
    \label{fig:placeholder}
\end{figure}

Foi realizada uma serie de testes com esta bancada, entre eles o de medição de forças aerodinâmicas em uma asa, de medição do empuxo dinâmico em motor e de medição do momento causado pelo acionamento de superfícies de comando (ailerons) em uma asa. O procedimento destes testes e seus resultados esta detalhado em [artigo_bancada_1]. Destacam-se aqui as principais conclusões levantadas por [artigo_bancada_1]:

\begin{itemize}
    \item Erro de X\% no CL
    \item ponto2
    \item ponto3
\end{itemize}

Fica claro portanto que o MVP foi um sucesso no sentido de que provou o funcionamento da bancada proposta em curto espaço de tempo, porém possuía problemas estruturais intrínsecos as escolhas para a prototipagem mecânica, e portanto não demonstrou fidelidade suficiente para seu uso. Além disso, ficaram claros diversos problemas do ponto de vista de execução, entre eles:

\begin{itemize}
    \item Aplicativo ainda em estagio inicial, apresentando uma interface pouco intuitiva
    \item Dificuldade para se ajustar o angulo de incidência do dispositivo testado
    \item Problemas recorrentes de mal-contato elétrico, resultando na perda de baterias inteiras de dados
    \item Pouco controle sobre o estado e funcionamento do computador da bancada
    \item Configuração do teste no computador da bancada era realizado modificando-se o script original, o que tornava esta configuração bastante suscetível a erros pelo operador 
    \item Software embarcado (no computador da bancada) pouco organizado, com as estruturas de funcionamento do software altamente entrelaçadas, dificultando sua expansão e manutenção
    \item Falta de sincronia entre os dados das células de carga com relação aos outros sensores, o que tornava o processamento dos dados um processo bastante massante
    \item Falta de informações sobre o teste no arquivo de gravação dos dados, o que fazia com que o operador precisasse recorrer a outros meios para guardar a informação sobre o que e como estava sendo feito em cada bateria de testes, causando muitas vezes a perda de informação sobre as mesmas
    \item Falta de software para analise dos dados, o que tornava o processo de utilização dos dados gerados pela bancada uma tarefa altamente especializada
\end{itemize}

Estes pontos foram tomados como base para a segunda versão da bancada, detalhada neste texto e denominada "Primeira versão final".

\section{Primeira versão final}

Levando em conta os pontos levantados no MVP deu-se inicio a modelagem e construção da primeira versão final da bancada.

\subsection{Modelo físico}

COMENTAR SOBRE O MODELO ESTATICO E A CORRECAO DE FALSA SUSTENTACAO

\subsection{Mecanica}

A estrutura da bancada utilizando perfis extrudados em alumínio foi modelada no software Solidworks.

\begin{figure}[!ht]
    \centering
    \includegraphics[width=.8\linewidth]{figuras/outras/placeholder.png}
    \caption{RENDERINGS DA ESTRUTURA DA BANCADA\cite{autor}.}
    \label{fig:estrutura_bancada}
\end{figure}

Os perfis de alumínio são adquiridos em barras de 1 metro de comprimento. A modelagem permitiu assim a definição das medidas dos cortes a serem realizados e a otimização do uso das barras.

Utilizando a estimativa de esforços assumida previamente foi desenvolvida a estrutura da bancada realizando-se simulações estruturais no software Ansys Mechanical de modo a se avaliar os deslocamentos máximos da estrutura. Para o projeto decidiu-se por não permitir deslocamentos verticais maiores que Xmm e deslocamentos horizontais maiores que Xmm. Esta restrição visa impedir que a deformação da bancada induza erros de alinhamento e consequentemente misture as medidas de sustentação e arrasto [X].

\begin{figure}[!ht]
    \centering
    \includegraphics[width=.8\linewidth]{figuras/outras/placeholder.png}
    \caption{RENDERING DA SIMULACAO DE ESFORÇOS MECANICOS NA BANCADA\cite{autor}.}
    \label{fig:simulacao_etsrutural}
\end{figure}

As conexões estaturais foram feitas com cantoneiras planas de aço, garantindo rigidez nas juntas, agregando fidelidade na simulação (uma vez que nela as juntas são consideradas rígidas).

Para o encaixe das células de carga na bancada, assim como dos suportes das guias lineares e dos \textit{pillow-blocks}, foram projetados adaptadores que posteriormente foram produzidos via manufatura aditiva de polímero. Este método de manufatura permitiu a rápida prototipagem dessas peças e acelerou o desenvolvimento do projeto.

\begin{figure}[!ht]
    \centering
    \includegraphics[width=.8\linewidth]{figuras/construcao/suporte_reforcado_celulas_3.jpg}
    \caption{Adaptadores para encaixe das células de carga de sustentação e momento\cite{autor}.}
    \label{fig:encaixe_celulas_sustentacao}
\end{figure}

\begin{figure}[!ht]
    \centering
    \includegraphics[width=.8\linewidth]{figuras/construcao/suporte_reforcado_celulas_1.jpg}
    \caption{Células de carga de sustentação e momento com adaptador\cite{autor}.}
    \label{fig:encaixe_celulas_sustentacao_2}
\end{figure}

\begin{figure}[!ht]
    \centering
    \includegraphics[width=.8\linewidth]{figuras/construcao/encaixe_pillow.jpg}
    \caption{Adaptador para encaixe do Pillow Block ("carrinho") na bancada\cite{autor}.}
    \label{fig:adaptador_pillow}
\end{figure}

\begin{figure}[!ht]
    \centering
    \includegraphics[width=.8\linewidth]{figuras/construcao/encaixe_sk12.jpg}
    \caption{Adaptador para encaixe dos suportes das guias lineares na bancada\cite{autor}.}
    \label{fig:adaptador_sk12}
\end{figure}

Devido a dificuldade na simulação de peças produzidas por este processo, que possuem alta anisotropia[X] e grande dispersão nos resultados dependendo da qualidade da impressão[X], foram realizados testes estruturais estáticos para se garantir que as peças não falhassem. 

Cabe aqui um adendo de que, para se garantir ainda maior rigidez a bancada como um todo, é de interesse a produção dessas peças em metal, realizando as devidas modificações para melhor se adequar ao processo de manufatura escolhido. A produção dessas peças em metal contudo não foi realizada dentro do tempo do presento trabalho.

\begin{figure}[!ht]
    \centering
    \includegraphics[width=.8\linewidth]{figuras/outras/placeholder.png}
    \caption{FOTO DA SOLUCAO COM AS GUIAS LINEARES - 3\cite{autor}.}
    \label{fig:placeholder}
\end{figure}

Dado que a bancada não teve seu movimento em X restrito exclusivamente por células de carga (como foi o caso no eixo Z), foi necessário permitir liberdade de movimento neste sentido, de modo que o carregamento se desse quase que exclusivamente na célula de carga. Para isto foram instaladas guias lineares nesta direção. Esta solução possui baixo atrito, além de apresentar pouquíssima folga no sentido transversal ao eixo da guia, o que favorece o alinhamento da bancada.

Como as guias possuem pouca folga, a tolerância de montagem também é pequena, isto é, uma pequena angulação entre os dois trilhos causaria travamento do sistema, o que é indesejado. Para permitir ajuste desse angulo durante a montagem das mesmas foi modelado um rasgo no furo das peças de encaixe dos trilhos, como mostrado na figura X. Assim pode-se correr as guias durante a instalação e ajustar o angulo de modo a garantir o não travamento.

\begin{figure}[!ht]
    \centering
    \includegraphics[width=.8\linewidth]{figuras/renders/suporte_sk12_com_rasgo.png}
    \caption{Detalhe do rasgo de parafuso para facilitar a montagem das guias  - 3\cite{autor}.}
    \label{fig:rasgo_suporte_sk12}
\end{figure}

A solução ideal para se garantir rigidez torcional da torre ao longo do eixo Z seria o treliçamento das laterais com os perfis de alumínio. Esta solução porem elevaria o custo do projeto para além do que a equipe possuía de verba disponível. A solução encontrada foi treliçar as duas laterais com cabos de aço.

\begin{figure}[!ht]
    \centering
    \includegraphics[width=.8\linewidth]{figuras/outras/placeholder.png}
    \caption{FOTO DA SOLUCAO COM OS CABOS DE ACO - 3\cite{autor}.}
    \label{fig:placeholder}
\end{figure}

Os furos dos quatro cantos da bancada foram rosqueados e argolas de fixação e cabos de aço foram instalados nas duas laterais, com tensionadores de modo a facilitar a instalação mesmo com pequenos desvios no comprimento dos cabos.

A primeira bancada utilizava barras roscadas como solução para o ajuste de angulo de incidência do componente testado. Esta solução possuía três problemas principais:

\begin{enumerate}
    \item O ajuste demorava, pois era necessário girar as duas barras até o angulo desejado medindo este angulo com o auxilio de um nível digital.
    \item Era difícil garantir que as duas barras estariam no mesmo angulo, isto é, que o componente não ficasse inclinado lateralmente.
    \item Pouca precisão se tinha no ajuste desse angulo.
\end{enumerate}

A solução encontrada foi trocar o ajuste continuo por um ajuste discreto, utilizando furos com angulação previamente projetada.

\begin{figure}[!ht]
    \centering
    \includegraphics[width=.8\linewidth]{figuras/renders/detalhe_ajuste_angulo_0_graus.png}
    \caption{Detalhe da peça de ajuste discreto de angulo de incidencia (angulo de 0 graus)\cite{autor}.}
    \label{fig:peca_angulo_0}
\end{figure}

\begin{figure}[!ht]
    \centering
    \includegraphics[width=.8\linewidth]{figuras/renders/detalhe_ajuste_angulo_18_graus.png}
    \caption{Detalhe da peça de ajuste discreto de angulo de incidencia (angulo de 18 graus)\cite{autor}.}
    \label{fig:peca_angulo_18}
\end{figure}

Esta solução garante que os mesmos ângulos sempre serão utilizados nos testes, o que adiciona facilidade no processamento posterior dos dados.

Os ângulos escolhidos foram: 0, 3, 6, 9, 12, 15 e 18 graus.

O principal problema desta solução é não permitir ajustes finos no angulo, o que a principio será um problema apenas caso se deseje descobrir o angulo de estol de asas. Este problema porem é tido como pequeno frente aos enfrentados com a solução anterior, e pode ser contornado fabricando-se uma peça de ajuste com mais furações e/ou uma peça de ajuste continuo a ser usada especificamente no teste de estol.
\subsection{Eletrônica}

Como um dos maiores problemas do MVP foi o de falha no sistema eletrônico devido a mal-contato, atenção especial foi dada a esta parte. Duas soluções foram propostas:

\begin{itemize}
    \item Produção de placas de circuito impresso para todos os sistemas
    \item Utilização de conectores mais robustos
\end{itemize}

Foram projetadas e produzidas duas PCBs: uma para os transdutores de célula de carga e outra para os transdutores de pressão diferencial, mostradas nas figuras \ref{projeto_pcb_pitot} e \ref{projeto_pcb_celulas}.

\begin{figure}[!ht]
    \centering
    \caption{Projeto e execução da placa de transdutores de pressão diferencial. Fonte: O autor.}
        \subfloat[Projeto da placa no software Autodesk Eagle.]{\includegraphics[width=0.4\columnwidth]{figuras/renders/pitot_board.png}}
        \label{projeto_pcb_pitot}
        \qquad
        \subfloat[Placa fresada e com sensores instalados.]{\includegraphics[width=0.4\columnwidth]{figuras/construcao/pcbs_montadas_2.jpg}}
        \label{placa_pcb_pitot}
\end{figure}

\begin{figure}[!ht]
    \centering
    \caption{Projeto e execução da placa de transdutores de pressão diferencial. Fonte: O autor.}
        \subfloat[Projeto da placa no software Autodesk Eagle.]{\includegraphics[width=0.4\columnwidth]{figuras/renders/loadcell_board.png}}
        \label{projeto_pcb_celulas}
        \qquad
        \subfloat[Placa fresada e com sensores instalados.]{\includegraphics[width=0.4\columnwidth]{figuras/construcao/pcbs_montadas_3.jpg}}
        \label{placa_pcb_celulas}
\end{figure}

Foram confeccionados carcaças para alocação de ambas as placas e seus conectores. Os conectores usados neste projeto foram do modelo GX16, popularmente conhecidos como "conectores de aviação". Estes possuem trava rosqueada e ótimo contato elétrico. Os conectores ficam presos no case e não na placa, de modo que em caso de acidentes de tracionamento dos cabos as placas não sejam danificadas. As versões finais das placas, já instaladas em seus respectivos cases são mostradas nas figuras \ref{case_pitot}a e \ref{case_celulas}b.

\begin{figure}[!ht]
    \centering
    \caption{Cases desenvolvidos para as placas de aquisição. Fonte: O autor.}
        \subfloat[Hardware para aquisição dos dados de velocidade e ângulo de ataque do escoamento.]{\includegraphics[width=0.4\columnwidth]{figuras/calibracao/caixa_pitot.jpg}}
        \label{case_pitot}
        \qquad
        \subfloat[Hardware para aquisição das cargas aerodinâmicas.]{\includegraphics[width=0.4\columnwidth]{figuras/calibracao/caixa_celulas.jpg}}
        \label{case_celulas}
\end{figure}

\subsection{Software Embarcado}

Como já mencionado, todo o software embarcado foi refatorado da versão MVP para esta primeira versão final. Alguns pontos principais merecem destaque:

\begin{itemize}
    \item Integração do Arduino como sensor na plataforma central, permitindo que mais Arduinos sejam conectados paralelamente de forma simplificada
    \item Implementação de nova rotina de ligação da bancada, adaptando automaticamente as configurações internas aos sensores que estiverem conectados, evitando a necessidade de alterar o código toda vez que um novo teste for executado
    \item Implementação de rotina de configuração do teste remotamente via app, permitindo que o operador inclua no arquivo de dados detalhes do teste como o dispositivo testado, seu ângulo de incidência e outras condições, facilitando o entendimento dos dados pelo responsável por processa-los 
    \item Inclusão de código para uso da sonda de ângulo de ataque através dos sensores de pressão diferencial
    \item Transferência de todas as configurações do software para um arquivo de configuração simplificada, permitindo que pessoas sem experiencia com o software embarcado consigam realizar configurações avançadas de forma simples
\end{itemize}

Sobre o primeiro ponto, na versão MVP o Arduino que fazia a aquisição do sinal das células não se comunicava com o Raspberry Pi (plataforma central). Deste modo, seus dados não possuíam sincronia, e esta deveria ser feita posteriormente por quem fosse se utilizar dos dados. Além de exigir um grande trabalho manual, este processo muitas vezes não ficava com qualidade satisfatória, e os dados acabavam não sincronizados. Para resolver este problema o Arduino passou a ser conectado na plataforma central e tratado por esta como um sensor, que responde a pedidos de envio de dados. Para garantir robustez na troca de dados foi implementado protocolo de comunicação de dois sentidos com confirmação de recebimento e checksum (valor calculado nos dois sistemas de forma independente, com base nos demais dados comunicados, de forma a validá-los).

\subsection{Bancada final}

As figuras \ref{fig:render_bancada_completa} e \ref{fig:bancada_construida} mostram a modelagem final da bancada e mesma construída.

\begin{figure}[!ht]
    \centering
    \includegraphics[width=.8\linewidth]{figuras/renders/perspectiva_sem_asa_com_pitot.png}
    \caption{Renderização da bancada modelada no software Solidworks 2016. Fonte: O autor.}
    \label{fig:render_bancada_completa}
\end{figure}

\begin{figure}[!ht]
    \centering
    \includegraphics[width=.7\linewidth]{figuras/calibracao/bancada_completa.jpg}
    \caption{Foto da bancada construída. Fonte: O autor.}
    \label{fig:bancada_construida}
\end{figure}

\section{Calibração}

\subsection{Calibração individual das células}
A calibração individual das células foi primeiramente realizada utilizando-se um sistemas de cordas e polias com pesos nas pontas (figuras \ref{polias_1}a e \ref{polias_1}b).

\begin{figure}[!ht]
    \centering
    \caption{Sistema de pesos e polias utilizado inicialmente para calibração das células. Fonte: O autor.}
        \subfloat[Polia utilizada para carregamento vertical.]{\includegraphics[width=0.4\columnwidth]{figuras/calibracao/polia1.jpg}}
        \label{polias_1}
        \qquad
        \subfloat[Peso utilizado para carregamento das células.]{\includegraphics[width=0.4\columnwidth]{figuras/calibracao/polia2.jpg}}
        \label{polias_2}
\end{figure}

Cada peso foi avaliado com uma balança de precisão com resolução de 1g e erro de +-0.5g, mostrada na figura \ref{fig:balanca_precisao}.

\begin{figure}[!ht]
    \centering
    \includegraphics[width=.4\linewidth]{figuras/calibracao/balanca_precisao.jpg}
    \caption{Balança de precisão utilizada para aferição dos pesos de calibração. Fonte: O autor.}
    \label{fig:balanca_precisao}
\end{figure}

Cada célula foi inicialmente carregada apenas com o balde onde os pesos eram colocados. As células eram então zeradas e o valor respectivo a 0g era então anotado.
Para cada nova carga imposta, um vetor de 100 pontos eram coletado, com a média, desvio padrão e coeficiente de variação dos mesmos extraída.

Notou-se já na calibração da primeira célula que o coeficiente de variação (CV) estava alto, chegando a mais de 4\%, e quando deixada a variar a medida demorava minutos até estabilizar. Foram então impostos carregamentos e descarregamentos sucessivos, e uma histerese com CV que chegava aos 18\% entre os diferentes ciclos estava presente. Esta histerese porém não se apresentava quando o sistema de polias não era utilizado. A solução encontrada foi realizar a calibração sem o sistema de polias, invertendo as células e carregando-as diretamente. Desta forma o CV máximo das 100 aquisições ficou abaixo de 0.2\% para todas as células, com as mesmas estabilizando em décimos de segundo. Os dados adquiridos se encontram na tabela \ref{tab:calib_celulas}.

\begin{table}[]
\centering
\resizebox{\textwidth}{!}{\begin{tabular}{llllllllllll}
Massa {[}g{]} & Peso {[}N{]} &  & \begin{tabular}[c]{@{}l@{}}Célula \\ Horizontal\end{tabular} & \begin{tabular}[c]{@{}l@{}}Célula\\ Frontal\\ Esquerda\end{tabular} & \begin{tabular}[c]{@{}l@{}}Célula\\ Frontal\\ Direita\end{tabular} & \begin{tabular}[c]{@{}l@{}}Célula\\ Traseira\\ Esquerda\end{tabular} & \begin{tabular}[c]{@{}l@{}}Célula\\ Traseira\\ Direita\end{tabular} &  & \begin{tabular}[c]{@{}l@{}}Arrasto\\ (max)\end{tabular} & \begin{tabular}[c]{@{}l@{}}Arrasto\\ (min)\end{tabular} & \begin{tabular}[c]{@{}l@{}}Arrasto\\ (medio)\end{tabular} \\
 &  &  &  &  &  &  &  &  &  &  &  \\
0 & 0.0 &  & -420 & -59 & 0 & 40 & -15 &  & 15128 & -14256 & 436 \\
142 & 1.4 &  & 62860 & 31544 & 31548 & 31580 & 31222 &  & 68497 & 45658 & 57078 \\
208 & 2.0 &  & 92192 & 46222 & 46266 & 46170 & 45652 &  & 101682 & 73246 & 87464 \\
352 & 3.5 &  & 155114 & 77814 & 77839 & 77632 & 76934 &  & 166984 & 147624 & 157304 \\
491 & 4.8 &  & 217264 & 109054 & 109067 & 108810 & 107894 &  & 237679 & 190734 & 214207 \\
1194 & 11.7 &  & 525814 & 264322 & 264342 & 263620 & 261004 &  & 499000 & 450000 & 474500 \\
1995 & 19.6 &  & 878342 & 441205 & 440885 & 439952 & 435844 &  & 831000 & 778000 & 804500 \\
3541 & 34.7 &  & 1562200 & 784432 & 784167 & 781870 & 775270 &  & 1580000 & 1382000 & 1481000 \\
6344 & 62.2 &  & 2794214 & 1404280 & 1403660 & 1400334 & 1387765 &  & 2770000 & 2532000 & 2651000 \\
9362 & 91.8 &  & 4099312 & 2063650 & 2061925 & 2058910 & 2043574 &  & 4032970 & 3916820 & 3974895
\end{tabular}}
\caption{Medições das massas de calibração e respectivas leituras das células.}
\label{tab:calib_celulas}
\end{table}

\begin{figure}[!ht]
    \centering
    \includegraphics[width=.8\linewidth]{figuras/calibracao/plot_erro_sustentacao.pdf}
    \caption{Erro relativo médio para medições das células de sustentação. Fonte: O autor.}
    \label{fig:curva_erros_sustentacao}
\end{figure}

Outro problema foi então encontrado nesta calibração. Mesmo sem o uso do sistema de polias, uma histerese que chegava a 40\% para baixas cargas e se mantinha perto dos 10\% para as cargas mais altas estava presente na medição de arrasto. Foi realizada uma tentativa de calibração da célula de arrasto instalando-a na vertical, não mais em série com as guias lineares, e a calibração da mesma não apresentou mais histerese, tendo resultados compatíveis com os das demais células.

A hipótese mais provável é de que a histerese seja resultado do atrito nas guias lineares, tornando portanto esta solução pior que a alternativa sem as guias. A mudança de uma solução para a outra porém exigiria compra de novas células, além da modificação da placa de aquisição de sinais, bem como de seu código, ambos adaptados para o uso de apenas uma célula de arrasto. Por este motivo não foi realizada a mudança na solução, mas a mesma é recomendada com máxima prioridade para uma modificação futura do projeto.

Para que o uso da bancada tenha maior fidelidade à realidade, a célula de arrasto foi recolocada em sua posição original e a calibração da mesma foi realizada nesta configuração (em série com as guias lineares), com a histerese presente. Como pode-se ver na figura \ref{fig:curva_erros_arrasto}, deve-se esperar desvios de até 22\% para cargas de até 1,2kg e de aproximadamente 6\% para cargas maiores.

\begin{figure}[!ht]
    \centering
    \includegraphics[width=.8\linewidth]{plots/celulas_plot_1.pdf}
    \caption{Erro relativo máximo e mínimo para medições de arrasto. Fonte: O autor.}
    \label{fig:curva_erros_arrasto}
\end{figure}

Foram então plotados os valores médios de leitura contra cada carregamento e uma regressão linear foi feita para cada célula. A função de transferência para uma das células de sustentação se encontra na figura \ref{fig:curva_celula_vertical}. Os coeficientes para todas as células se encontram na tabela \ref{tab:coeficientes_celulas}.

\begin{figure}[!ht]
    \centering
    \includegraphics[width=.8\linewidth]{figuras/calibracao/plot_regressao_sustentacao.pdf}
    \caption{Pontos adquiridos e curva calibrada para uma das células de medição de forças verticais. Fonte: O autor.}
    \label{fig:curva_celula_vertical}
\end{figure}


\begin{table}[]
\centering
\resizebox{\textwidth}{!}{\begin{tabular}{lllllllllll}
 &  & \begin{tabular}[c]{@{}l@{}}Célula\\ Horizontal\end{tabular} & \begin{tabular}[c]{@{}l@{}}Célula\\ Frontal\\ Esquerda\end{tabular} & \begin{tabular}[c]{@{}l@{}}Célula\\ Frontal\\ Direita\end{tabular} & \begin{tabular}[c]{@{}l@{}}Célula\\ Traseira\\ Esquerda\end{tabular} & \begin{tabular}[c]{@{}l@{}}Célula\\ Traseira\\ Direita\end{tabular} &  & \begin{tabular}[c]{@{}l@{}}Arrasto\\ (max)\end{tabular} & \begin{tabular}[c]{@{}l@{}}Arrasto\\ (min)\end{tabular} & \begin{tabular}[c]{@{}l@{}}Arrasto\\ (medio)\end{tabular} \\
Coef. Angular &  & 44708 & 22496 & 22479 & 22439 & 22264 &  & 44006 & 42192 & 43099 \\
Coef. Linear &  & 2365 & 796 & 914 & 748 & 432 &  & 10243 & -27835 & -8796
\end{tabular}}
\caption{Coeficientes de calibração das células}
\label{tab:coeficientes_celulas}
\end{table}

Foi também plotado o erro das curvas estimadas por regressão contra os valores do teste, obtendo o erro relativo a cada carregamento. A regressão utilizando apenas o coeficiente angular apresentou diferença menor que 0.01\% em relação a regressão com o coeficiente linear também presente, de modo que a primeira forma de regressão foi utilizada. A vantagem desta opção é diminuir o volume de dados de calibração passados para o código da bancada.

Como pode ser visto, na faixa de medição para baixos pesos o desvio relativo da medida é maior, mas ainda bastante baixo, com teto para as células de sustentação de 1.06\% para valores inferiores a 1,2kg e 0.59\% para valores superiores. O erro médio de todas as medições foi de 0.4\%.

% \subsection{Calibração integrada das células}

% Em seguida foi realizada a calibração conjunta de cargas. Foram impostas diversas combinações de carga de arrasto e sustentação de forma simultânea visando avaliar o erro devido ao acoplamento das cargas, isto é, o quanto uma carga influencia na leitura da outra. Ums proposta de correçao para este acoplamento seria realizada segundo a equação X.

% \begin{equation}
%     L = A1*L_{Raw} - A2*D_{Raw}
% \end{equation}

% \begin{equation}
%     D = A3*D_{Raw} - A4*L_{Raw}
% \end{equation}

% Os resultados para as curvas com e sem cargas simultaneas se encontram plotados nas figuras X e X.

% \begin{figure}[!ht]
%     \centering
%     \includegraphics[width=.8\linewidth]{figuras/outras/placeholder.png}
%     \caption{Curva de sustentaçao com e sem carga de arrasto. Fonte: O autor.}
%     \label{fig:placeholder}
% \end{figure}

% Pode-se verificar para o caso da mediçao de sustentaçao que existe um desvio que fica entre 0.3\% e 1.74\% entre as curvas medidas com e sem cargas simultâneas, indicando que de fato existe um acoplamento entre as medições. Para o caso da leitura de arrasto, porém, o desvio devido a histerese é uma ordem de grandeza maior que os efeitos esperados para o acoplamento, mascarando os resultados do mesmo. Deste modo, a correçao por acoplamento ficou bastante comprometida, com resultados que nao trariam ganho em precisao, e portanto foi descartada.

\subsection{Calibração do tubo de Pitot e sonda de AoA}

\begin{figure}[!ht]
    \centering
    \includegraphics[width=.8\linewidth]{figuras/outras/5holepitot.png}
    \caption{Sonda de ângulo de ataque de 5 tomadas. A tomada numero 6, de pressão estática, é posicionada lateralmente ao tubo. Fonte: \cite{lee1986calibration}}
    \label{fig:5_hole_probe}
\end{figure}

Dada a maior flexibilidade e pouco aumento de custo, foram utilizados apenas Pitots de múltiplas tomadas na bancada, ignorando-se as tomadas diferenciais quando da não necessidade de medição de ângulo de ataque. 

As calibrações dos tubos de Pitot de múltiplas tomadas foram realizadas no túnel de vento da UFSC variando-se discretamente o ângulo de ataque de -20 a 20 graus, com passo de 2 graus, utilizando o disco de medição angular presente na balança do túnel de vento. O teste foi realizado para 7 velocidades de escoamento no túnel, apresentadas ao final deste capitulo. 

\begin{figure}[!ht]
    \centering
    \caption{Sistema utilizado para calibração da sonda de ângulo de ataque. Fonte: O autor.}
        \subfloat[Sonda instalada em adaptador para o túnel de vento, com caixa da IMU acoplada.]{\includegraphics[width=0.4\columnwidth]{figuras/calibracao/dispositivo_sonda.jpg}}
        \label{sonda_tunel_1}
        \qquad
        \subfloat[Sistema instalado no túnel de vento.]{\includegraphics[width=0.4\columnwidth]{figuras/calibracao/sonda_tunel.jpg}}
        \label{sonda_tunel_2}
\end{figure}

Para aumentar a agilidade no teste e no pós-processamento dos dados foi instalada uma IMU na sonda, medindo-se assim o ângulo de ataque da mesma através da equação \ref{eq:acc_angle}. Foram realizadas medições de pressão diferencial nos pares de tomadas 1-6 e 4-5, denominadas respectivamente pressão-dinâmica e pressão-diferencial.

\begin{equation}
    \alpha = \arctan \frac{Acc_y}{\sqrt{Acc_x^2+Acc_y^2}}   
    \label{eq:acc_angle}
\end{equation}


Segundo \cite{borges2008desenvolvimento} o ângulo do escoamento é proporcional à razão $\frac{\Delta P}{q}$ onde $\Delta P$ é a diferença de pressão entre as tomadas alinhadas com o plano de medição (picada ou guinada) e $q$ é a pressão dinâmica do escoamento.

% Para a normalização dos dados \cite{lee1986calibration} recomenda o referenciamento em relação a media das pressões nas quatro tomadas laterais (2, 3, 4, 5), conforme as equações X, Y e Z.

% \begin{equation}
%     C_{P_{pitch}} = \frac{P_4 - P_5}{P_1 - P_{ref}}
% \end{equation}
% \begin{equation}
%     C_{P_{total}} = \frac{P_1 - P_{total}}{P_1 - P_{ref}}
% \end{equation}
% \begin{equation}
%     C_{P_{static}} = \frac{P_{ref} - P_{static}}{P_1 - P_{ref}}
% \end{equation}
% \begin{equation}
%     P_{ref} = \frac{P_2 + P_3 + P_4 + P_5}{4}
% \end{equation}

Para o presente caso as tomadas de yaw (2 e 3) nao foram utilizadas, dado que o interesse é apenas na mediçao do ângulo de ataque.

% A pressão estática é medida por um transdutor de pressão atmosférica, presente na IMU, enquanto a pressão total do escoamento é a soma da pressão-dinamica (diferença das tomadas 1 e 6) quando $\alpha$ é igual a 0 graus com a pressão estática.

Os resultados da medição de $\frac{\Delta P}{q}$ por $\alpha$ se encontram na figura \ref{fig:Deltapq_alpha_plot}.

\begin{figure}[!ht]
    \centering
    \includegraphics[width=.7\linewidth]{figuras/calibracao/dp-q_plot.pdf}
    \caption{Medições de $\frac{\Delta P}{q}$ por $\alpha$ no teste em túnel de vento. Neste gráfico se encontram presentes os pontos adquiridos para todas as velocidades de escoamento testadas no túnel de evento. Fonte: O autor.}
    \label{fig:Deltapq_alpha_plot}
\end{figure}

Foi realizada uma regressão linear neste gráfico \citep{borges2008desenvolvimento}. O erro máximo da linearização e o erro médio quadrático para ângulos entre -15 e +15 graus são respectivamente 1.32 graus e 0.42 graus.

\begin{equation}
    \alpha = 2.34 + 15.45 * \frac{\Delta P}{q}
\end{equation}

O modulo do vetor velocidade pode então ser calculado pela equação \ref{eq:velocity_q}. 

\begin{equation}
    U = \sqrt{\frac{2q}{\rho}}
    \label{eq:velocity_q}
\end{equation}

% TABELA COM CONDICOES ATMOSFERICAS
% PRESSAO 101.8kPa
% Temp 16C
% Umidade 70\%
% Densidade estimada 1.227

Para averiguação do valor verdadeiro da velocidade do escoamento, tomado como referencia para caibração, foi utilizado um anemômetro de fio quente. O gráfico \ref{fig:U_alpha} mostra o valor verdadeiro da velocidade e a velocidade medida pela sonda, com variação do ângulos de ataque de -20 a 20 graus.

\begin{figure}[!ht]
    \centering
    \includegraphics[width=.8\linewidth]{figuras/calibracao/u_ver_plot.pdf}
    \caption{Velocidade medida através da sonda $vs$ velocidade medida por anemômetro de fio quente. Fonte: O autor.}
    \label{fig:U_alpha}
\end{figure}

O erro máximo da velocidade medida e o erro médio quadrático para cada velocidade de escoamento do túnel de vento, para ângulos entre -15 e +15 graus foram de respectivamente 12.4\% e 1.8\%.

\section{Testes}

Uma serie de testes foi realizada inicialmente de forma mais rápida visando-se buscar melhorias a serem feitas na bancada, seja na parte mecânica, eletrônica ou de software. Estes testes se deram simultaneamente ao projeto e auxiliariam em diversas melhorias do projeto e do processo de execução dos testes, entre eles:

\begin{enumerate}
    \item Verificação da necessidade de simulação da bancada para otimização da posição do tubo de Pitot.
    \item Verificação da calibração das células de carga e tubos de Pitot.
    \item Inclusão da possibilidade de zeragem das células e Pitots ao inicio do teste via app
    \item Inclusão da IMU para medição das acelerações sofridas durante o teste.
    \item Verificação da necessidade de uso da sonda acoplada ao componente a ser testado
\end{enumerate}

Uma vez finalizada a Bancada V1, foi iniciada a condução dos testes comparativos.

\subsection{O componente de teste}

Os testes comparativos tem por finalidade gerar dados sobre geometrias conhecidas que possam ser confrontados com literatura existente. Para este teste o componente ideal seria uma asa utilizando perfis do tipo NACA, que possuem vasta literatura de dados experimentais, porém devido a problemas orçamentários e de tempo não foi possível a construção de tal componente, sendo então os testes da bancada realizados com a 1a aeronave protótipo da equipe para o ano de 2019. As principais especificações geométricas desta aeronave se encontram em planilha anexada.

Foi escolhida a velocidade de cruzeiro da aeronave para carga total, 12 m/s, como velocidade dos testes. O Reynolds referente a Corda Média Aerodinâmica da asa é de aproximadamente 290.000 para as condições do teste.

\begin{figure}[!ht]
    \centering
    \includegraphics[width=.6\linewidth]{figuras/renders/aviao_sozinho.png}
    \caption{Modelagem da aeronave Céu Azul 2019P1, utilizado nos testes deste trabalho. Fonte: O autor.}
    \label{fig:aviao_renderizado}
\end{figure}

\begin{figure}[!ht]
    \centering
    \includegraphics[width=.6\linewidth]{figuras/testes/prototipo_construido.JPG}
    \caption{Foto da aeronave Céu Azul 2019P1, utilizado nos testes deste trabalho. Fonte: O autor.}
    \label{fig:aviao_construido}
\end{figure}

Para a simulação de tal aeronave foi utilizado código de analise ADR, desenvolvido pela equipe Céu Azul Aeronaves com coordenação dO autor.. Este código utiliza para analise aerodinâmica o software de código-aberto AVL, que implementa o algoritmo Vortex Lattice Method, e faz correções de arrasto parasita para cada componente em separado. As figuras \ref{fig:adr_resultados_1} e \ref{fig:adr_resultados_2} mostra as curvas aerodinâmicas simuladas para tal aeronave.

% A figura \ref{fig:avl_simulacao} mostra a simulação de uma das asas da aeronave de teste no software AVL.

% \begin{figure}[!ht]
%     \centering
%     \includegraphics[width=.8\linewidth]{figuras/outras/placeholder.png}
%     \caption{RENDERING DA SIMULACAO NO AVL. Fonte: O autor.}
%     \label{fig:avl_simulacao}
% \end{figure}

\begin{figure}[!ht]
    \centering
    \includegraphics[width=1\linewidth]{figuras/ADR/cl_cd_polar.png}
    \caption{Graficos de $C_L \times \alpha$, $C_D \times \alpha$ e Polar de Arrasto para a aeronave testada. Fonte: O autor.}
    \label{fig:adr_resultados_1}
\end{figure}

\begin{figure}[!ht]
    \centering
    \includegraphics[width=.5\linewidth]{figuras/ADR/cm_alpha_0.png}
    \caption{Gráfico de $C_M \times \alpha$ para a aeronave testada, considerando profundor com 0 graus de ângulo de incidência, como utilizado no teste. Fonte: O autor.}
    \label{fig:adr_resultados_2}
\end{figure}

\subsection{O teste}

Os testes foram conduzidos em uma avenida reta e comprida (Avenida Beira-mar Norte), essencialmente plana e em horário de trânsito reduzido (durante a madrugada).

\begin{figure}[!ht]
    \centering
    \includegraphics[width=.5\linewidth]{figuras/internet/here_open_street_maps_setas.png}
    \caption{Local onde foram realizados os testes. Fonte: O autor.}
    \label{fig:mapa_teste}
\end{figure}

O teste foi conduzido acelerando-se o carro até uma velocidade de aproximadamente 12m/s (43km/h) e mantendo essa velocidade durante o maior tempo possível. Sendo conduzido desta maneira o teste entrega resultados para apenas um valor de Reynolds. Foi assim feito pois em testes preliminares verificou-se que cada bateria de testes demorava um tempo considerável, e repetir ela para uma grande faixa de Reynolds reduziria o número de dados para cada velocidade. Decidiu-se portanto ter um maior numero de dados para um mesmo Reynolds, garantindo-se uma qualidade melhor dos resultados finais. Com mais tempo porem, ou uma condução mais rápida dos testes, poder-se-ia realizar os testes para mais valores de Reynolds.  

\begin{figure}[!ht]
    \centering
    \includegraphics[width=.4\linewidth]{figuras/calibracao/app_celular.jpg}
    \caption{Aplicativo desenvolvido rodando em celular Android com transmissor de radio conectado. Fonte: O autor.}
    \label{fig:app_celular}
\end{figure}

Para se averiguar a velocidade foram utilizados dois recursos: velocímetro do próprio carro e velocidade medida pelo pelos tubos de Pitot. Idealmente as duas medidas devem apresentar o mesmo resultado. Para se acompanhar a velocidade medida pelos Pitots em tempo real foi utilizado o aplicativo desenvolvido.

\begin{figure}[!ht]
    \centering
    \includegraphics[width=.5\linewidth]{figuras/testes/teste_aviao_completo.JPG}
    \caption{Foto do teste com aeronave Céu Azul 2019 P1. Fonte: O autor.}
    \label{fig:foto_teste_aviao}
\end{figure}

Para cada bateria de testes a asa foi instalada em um ângulo de incidência diferente. Devido a dificuldades técnicas, foram realizados testes apenas para os ângulos de 0, 3 e 6 graus. Para cada um dos ângulos, três baterias de teste foram conduzidas, buscando-se criar uma base de dados grande e diminuir a influencia dos testes nos resultados.


\chapter{Resultados}\label{chp:res}

O conjunto de dados, originais e derivados (aqueles criados a partir de dados originais), estão apresentados na tabela X.

\begin{table}[!ht]
    \centering
    \includegraphics[width=.8\linewidth]{figuras/outras/placeholder.png}
    \caption{TABELA COM DADOS GRAVADOS\cite{autor}.}
    \label{fig:placeholder}
\end{table}

Na imagem X pode ser visto um exemplo dos dados adquiridos durante uma bateria de testes. A gravação dos dados na plataforma ocorre a uma taxa fixa de X Hz, diferente da taxa original de aquisição de cada sensor. 

\begin{table}[!ht]
    \centering
    \includegraphics[width=.8\linewidth]{figuras/outras/placeholder.png}
    \caption{GRAFICOS COM FREQUENCIA DE AQUISICAO DE CADA SENSOR (SINAL DIGITAL)\cite{autor}.}
    \label{fig:placeholder}
\end{table}

Isso é feito de modo que todos apresentem uma mesma frequência de ocorrência na plataforma e portanto tenham sempre correspondência nos dados dos outros sensores, facilitando posterior processamento. Neste nivelamento alguns dados, por possuírem frequência de aquisição mais baixa que a de gravação da plataforma, acabam se repetindo até que novos valores estejam disponíveis.

\begin{figure}[!ht]
    \centering
    \includegraphics[width=.8\linewidth]{figuras/outras/placeholder.png}
    \caption{PLOTS OCUPANDO A PAGINA INTEIRA - 1\cite{autor}.}
    \label{fig:placeholder}
\end{figure}

\begin{figure}[!ht]
    \centering
    \includegraphics[width=.8\linewidth]{figuras/outras/placeholder.png}
    \caption{PLOTS OCUPANDO A PAGINA INTEIRA - 2\cite{autor}.}
    \label{fig:placeholder}
\end{figure}

\begin{figure}[!ht]
    \centering
    \includegraphics[width=.8\linewidth]{figuras/outras/placeholder.png}
    \caption{PLOTS OCUPANDO A PAGINA INTEIRA - 3\cite{autor}.}
    \label{fig:placeholder}
\end{figure}

\section{Limpeza, filtragem e redução dos dados}

\subsection{Limpeza de dados}

A premissa principal para o uso dos dados do teste é o de que os valores medidos estejam dentro do intervalo de uso dos sensores e de que as acelerações sobre o sistema sejam baixas. O intervalo de uso dos sensores se encontra na tabela X.

\begin{table}[!ht]
    \centering
    \includegraphics[width=.8\linewidth]{figuras/outras/placeholder.png}
    \caption{TABELA COM ZONA DE USO DE CADA SENSOR\cite{autor}.}
    \label{fig:placeholder}
\end{table}

O principal sensor a limitar o uso dos dados foi o Pitot. Abaixo de X m/s o sensor é incapaz de medir a velocidade corretamente, portanto todos os dados adquiridos abaixo dessa velocidade foram descartados, como pode ser visto em exemplo na figura X.

\begin{figure}[!ht]
    \centering
    \includegraphics[width=.8\linewidth]{figuras/outras/placeholder.png}
    \caption{PLOTS DE VELOCIDADE SUSTENTACAO E ACELERACAO Z\cite{autor}.}
    \label{fig:placeholder}
\end{figure}

Os dados resultantes desse processo são considerados validos, ainda que possam não ser interessantes para a medição por não se encaixarem no modelo proposto (i.e se forem medidos sob altas acelerações). Dados com aceleração em X, Y ou Z maior que 0.5g foram descartados no próximo sub-set.

Ainda que os dados resultantes desses dois processos sejam validos e utilizáveis no modelo, os mesmos podem não se encontrar na faixa de Reynolds estipulada para o teste. Dado que o teste foi executado a uma velocidade praticamente constante, o resultado final sera dado para apenas um valor de Reynolds, sendo assim, dados adquiridos que sejam validos mas não estejam próximos ao Reynolds do teste não devem ser utilizados, dado que devem apresentar comportamento aerodinâmico diferente daquele correspondente ao Reynolds desejado[X]. Desta forma os dados finais a serem utilizados para medição serão aqueles correspondentes aos instantes onde foi alcançada velocidade praticamente constante e com um máximo de 10\% de desvio com relação ao Reynolds desejado estipulado.

\begin{figure}[!ht]
    \centering
    \includegraphics[width=.8\linewidth]{figuras/outras/placeholder.png}
    \caption{PLOTS COM PATAMARES DE VELOCIDADE CONSTANTE\cite{autor}.}
    \label{fig:placeholder}
\end{figure}

\subsection{Filtragem}

Apos a limpeza dos dados, onde foram removidos aqueles que estavam fora da zona de interesse, buscou-se avaliar a filtragem do sinal, visando aumentar a relação sinal-ruido. Para isto foi realizada uma analise FFT[X] nos dados de carga, velocidade e aceleração.

\begin{figure}[!ht]
    \centering
    \includegraphics[width=.8\linewidth]{figuras/outras/placeholder.png}
    \caption{ANALISES FFT\cite{autor}.}
    \label{fig:placeholder}
\end{figure}

A partir do resultado das analises foram dimensionados filtros do tipo passa-baixa para a remoção dos ruídos de alta frequência em cada um dos dados[X]. Os resultados se encontram na figura X.

\begin{figure}[!ht]
    \centering
    \includegraphics[width=.8\linewidth]{figuras/outras/placeholder.png}
    \caption{DADOS APOS PASSA-BAIXA\cite{autor}.}
    \label{fig:placeholder}
\end{figure}

\subsection{Redução}

Em posse dos dados filtrados passa-se a etapa de redução. Não é do interesse do projetista a principio saber qual sustentação, arrasto ou momento foram alcançados durante os testes, pois esses são dados absolutos que dependem do tamanho do componente e da velocidade no instante medido[X]. Para facilidade de analise e mais justa comparação entre diferentes componentes é de interesse a redução destes dados na forma dos coeficientes aerodinâmicos[X].

EQUACAO DOS COEFICIENTES AERODINÂMICOS

As equações XYZ foram aplicadas a todas as amostras de dados gerando três novas colunas: CL, CD e CM. Para cada bateria de teste foi calculado o valor médio de cada coeficiente, e em seguida foi calculado para cada angulo de incidência o valor médio dos coeficientes das baterias, assim como seu desvio padrão.

Os valores finais de media e desvio padrão para os coeficientes de cada angulo de incidência estão explicitados nos gráficos X Y e Z.

\begin{figure}[!ht]
    \centering
    \includegraphics[width=.8\linewidth]{figuras/outras/placeholder.png}
    \caption{GRAFICOS CX VS ALPHA COM DESVIO PADRAO\cite{autor}.}
    \label{fig:placeholder}
\end{figure}

DISCUTIR RESULTADOS E AVALIAR A INCERTEZA DA BANCADA

 % Texto do cap4.tex
\chapter{Conclusão}\label{chp:conc}

    Este trabalho procurou descrever o projeto, construção e teste de uma bancada experimental para caracterização aerodinâmica de componentes de VANTs, visando eliminar a necessidade de testes em túnel de vento ou ensaios em voo.
    
    O trabalho foi desenvolvida no âmbito de uma equipe de projeto aeronáutico, buscando preencher a necessidade da mesma por dados experimentais que norteassem a validação e otimização de seus projetos.
    
    O trabalho foi bem sucedido no sentido de projetar, construir e testar tal bancada, cumprindo os requisitos estabelecidos de custo, facilidade de uso, desenvolvimento de softwares de suporte e documentação. 
    
    Os resultados obtidos no teste estão dentro do esperado quando comparados as simulações para os mesmos componentes testados.  Não foram testados contudo componentes com valores experimentais conhecidos, de forma que não se pode dizer que a bancada esta validada. Esperava-se ao inicio deste trabalho validar ou invalidar a bancada, porém a necessidade de finalizar os ensaios experimentais antes do previsto impediu que tal ação fosse concluída.
    
    Durante o desenvolvimento deste trabalho, dificuldades surgiram onde se poderia e onde não se poderia imaginar. Atrasos na entrega dos componentes, cargas não previstas, erros na execução dos testes, comportamentos inesperados do software de aquisição de dados, comportamentos inesperados dos sensores, todos foram problemas encontrados durante o caminho, felizmente resolvidos, um a um, ainda que atrasando o cronograma previsto.
    
    O uso da bancada hoje se encontra bastante facilitado, sendo possível realizar os testes e a analise dos resultados em uma noite de trabalho, considerando um bom planejamento e execução dos mesmos.
    
\section{Trabalhos futuros}
    
    Pode-se dizer que este trabalho se encontra em estagio avançado, mas ainda não concluído. A principal necessidade que se enxerga para um trabalho futuro é a validação da mesma para componentes já experimentalmente caracterizados. Esta validação agregaria em muito e permitiria que a equipe Céu Azul passasse a realizar seus ensaios experimentais visando otimizar o projeto com uma maior garantia sobre a validade dos resultados. Além da própria validação, alguns pontos a se tratar no futuro, seja por este autor ou por um terceiro, são levantados:
    
\begin{itemize}
    \item Mudança na solução da bandeja inferior da balança, passando a utilizar 3 células de carga e eliminando as guias lineares.
    \item Calibração dos fatores de acoplamento entre medições de sustentação e arrasto
    \item Validação da bancada contra geometrias com valores experimentais já consolidados, tais como esferas, cilindros e asas de perfil NACA
    \item Desenvolvimento de interface gráfica para carregamento dos dados de testes, visando eliminar a necessidade de execução das analises via scripts.
    \item Inclusão de rotina de calibração das células de carga e dos tubos de Pitot via app, tornando a calibração um processo rápido e facilitado.
    \item Inclusão de rotina de testes de forma guiada pelo app, isto é, em um formato de passo-a-passo, com instruções de execução, avisos de mal-uso ou resultados inconsistentes e automatização das analises.
    \item Possibilidade de transferência dos dados da plataforma de aquisição para o celular via Bluetooth, eliminando a necessidade de retirada do cartão de memoria da mesma.
    \item Inclusão de modo de gravação de áudio no app, visando a possibilidade de uma "narração" do teste pelo responsável. Este áudio permitiria que avaliações extras pudessem ser realizadas pelo responsável pela analise, como identificação de possíveis erros de execução, anormalidades ou mesmo testes diferenciados, como a execução de comandos na aeronave a serem posteriormente caracterizados. 
\end{itemize}
 % Texto do cap5.tex



%%%%%%%%%%%%%%%%%%%%%%%%%%
%%Elementos Pós-Textuais%%
%%%%%%%%%%%%%%%%%%%%%%%%%%
%%--------Bibliografia--------
\Bibliografia{referenciasbibliograficas} % arquivo bibtex com as referências (referenciasbibliograficas.bib)

%%--------Apêndice--------
%% Descomente caso exista a necessidade da inclusão de apêndices
%\apendice % Comando que inclui os apêndices
%\include{7_Apendice/apendice_1}   % Arquivo apendice_1.tex
%\include{7_Apendice/apendice_2}   % Arquivo apendice_2.tex
%%---------------------------------------------------------------

%%--------Anexos-------------------------------------------------
%% Descomente caso exista a necessidade da inclusão de anexos
%\anexo % Comando que inclui os anexos (outros autores)
%\include{8_Anexos/anexo_1}  % Arquivo anexo_1.tex
%\include{8_Anexos/anexo_2}  % Arquivo anexo_2.tex
%%---------------------------------------------------------------

\end{document}
