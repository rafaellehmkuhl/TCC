\chapter{Metodologia}\label{chp:met}

\section{Projeto}

\begin{itemize}
    \item Levantamento das necessidades
    \item Levantamento das possiveis solucoes
    \item Projeto da bancada 1.0
    \item Levantamento dos problemas da 1.0
    \item Levantamento das possiveis soluçoes para a 1.0
\end{itemize}

\subsection{Necessidades}

O cliente do presente projeto é a equipe Céu Azul Aeronaves, que representa anualmente a UFSC na competiçao SAE Brasil Aerodesign.

O objetivo da bancada proposta neste trabalho é o de medir os coeficientes aerodinamicos (de sustentaçao, arrasto e momento) em diversas geometrias expostas a um escoamento de ar relativamente bem comportado. Tais geometrias incluem componentes de aeronaves do tipo VANT, como asas, profundores, fuselagens e outros dispositivos. Além disso, o uso para a mediçao de curvas de empuxo dinamico (isto é, o empuxo medido para diferentes velocidades) de motores é desejado.

Tal bancada deve possuir rigidez suficiente para que suas mediçoes sejam consistentes no tempo. Deve ser possivel também prende-la a caçamba de uma camionete ou ao rack de um carro comum, de modo a facilitar sua instalaçao sem grande dificuldade.

Levando em conta que a equipe sofre de alta rotatividade de membros, e nem sempre é possivel garantir que existirao pessoas capacitadas a trabalhar neste projeto, é importante que a bancada em questao seja entregue na forma de um "produto final", isto é, funcione sem a necessidade de conhecimento de suas intricidades, com pouco mais que um manual de instruçoes e consideraçoes para seu uso.

Da mesma forma que o uso da bancada deve ser facilitado, também deve ser assim o uso dos dados fornecidos por ela. Idealmente se deseja que a mesma entregue os coeficientes finais, ja processados, assim como suas incertezas de mediçao.

\subsection{Consideraçoes e estimativas}

Assume-se para o presente projeto que o escoamento de ar que alcança a geometria analisada tem direçao relativamente estavel (e aproximadamente paralela ao deslocamento do veiculo) e velocidade praticamente constante para pequenos trechos a serem analisados. Oscilaçoes da ordem de 10\% na velocidade sao esperadas e podem ser facilmente mensuradas através de sensoriamento adequado (tupo de pitot e GPS, por exemplo). 

\begin{figure}[!ht]
    \centering
    \includegraphics[width=.8\linewidth]{figuras/placeholder.png}
    \caption{FIGURA COM ANGULO DE ESCOAMENTO INDUZIDO PELO VEICULO\cite{autor}.}
    \label{fig:vehicle-angle}
\end{figure}

É possivel também que a carroceria do veiculo induza um angulo diferente de zero ao escoamento que chega a bancada (ver \prettyref{fig:vehicle-angle}). Se este angulo for de pequena ordem (menor que 15 graus) um pitot de multiplas tomadas pode ser utilizado para medi-lo e normalizar os dados em um pos-processamento. Grandes desvios de direçao ou velocidades muito instaveis provavelmente tornarao o processamento dos dados dos testes impeditivo.

\begin{figure}[!ht]
    \centering
    \includegraphics[width=.8\linewidth]{figuras/placeholder.png}
    \caption{FIGURA MOSTRANDO FORCA RADIAL INDUZIDA POR PISTA CURVA\cite{autor}.}
    \label{fig:placeholder}
\end{figure}

\begin{figure}[!ht]
    \centering
    \includegraphics[width=.8\linewidth]{figuras/placeholder.png}
    \caption{FIGURA MOSTRANDO FATOR DE CARGA SENTIDO EM OSCILACOES\cite{autor}.}
    \label{fig:placeholder}
\end{figure}

Assume-se também que existem obstaculos na pista de teste (tais como buracos e pequenas elevaçoes), mas que de forma geral o teste sera conduzido em pistas relativamente retas e planas, de modo que nao existam vibraçoes constantes ou aceleraçoes radiais a serem modeladas no sistema. Pequenas oscilaçoes podem ser medidas por dispositivos adequados e levadas em conta no pos-processamento.

Para os projetos mecanico e eletronico considera-se as possiveis situaçoes a serem mensuradas:


\begin{itemize}
    \item Caracterizaçao de aeronave completa com 200N de sustentaçao, 50N de arrasto e XXXN de Momento
    \item Caracterizaçao de conjunto motopropulsor com 60N de empuxo estatico
    \item Escoamentos com velocidade de até 25m/s
    \item Fatores de carga na bancada de até 2g
\end{itemize}

A camionete a ser utilizada nos testes é do modelo Fiat Strada 2010, tendo a caçamba uma altura de XXXm e o teto do carro elevando-se XXXm além da altura da caçamba.

\subsection{Soluçao proposta}

\begin{figure}[!ht]
    \centering
    \includegraphics[width=.8\linewidth]{figuras/placeholder.png}
    \caption{ESQUEMATICO DA BANCADA + TORRE\cite{autor}.}
    \label{fig:placeholder}
\end{figure}

A soluçao proposta consiste em uma bancada sensoriada montada sobre uma torre que a coloque na altura desejada. As forças na bancada sao sensoriadas com quatro celulas de cargas verticais e uma ou mais celulas de carga horizontais. As celulas horizontais permitem a mediçao do arrasto, o somatorio das celulas verticais permitem a mediçao da sustentaçao, a diferença entre as celulas frontais e traseiras permite a mediçao do momento de picada (pitch) e a diferença entre as celulas da esquerda e da direita permite a mediçao do momento de rolagem (roll).

\begin{figure}[!ht]
    \centering
    \includegraphics[width=.8\linewidth]{figuras/placeholder.png}
    \caption{ESQUEMATICO COM DISPOSICAO DAS CELULAS\cite{autor}.}
    \label{fig:placeholder}
\end{figure}

Ainda na bancada serao instalados ao menos um tubo de pitot e um pitot de multiplas tomadas. Ambos serao colocados em escoamento livre, proximos ao teto da bancada, de modo a se aproximar da altura da geometria a ser analisada. O sistema deve comportar ainda a adiçao de novos sensores de forma a se caracterizar o escoamento em outros pontos de interesse. Para a mediçao da velocidade absoluta do carro sera utilizado um sistema GNSS acoplado a bancada.

Para a mediçao do fator de carga vertical, assim como da temperatura e pressao do ar, uma IMU com acelerometro, barometro e termometro sera instalada bancada.

\begin{figure}[!ht]
    \centering
    \includegraphics[width=.8\linewidth]{figuras/placeholder.png}
    \caption{FOTO DA IMU\cite{autor}.}
    \label{fig:placeholder}
\end{figure}

Num primeiro momento foi levantada a possibilidade de se utilizar um computador portatil como central de comando da bancada, porém isto poderia tornar o uso dificultado, ja que em geral estes computadores possuem baterias que permitem poucas horas de uso continuo, limitando o tempo de execuçao de teste, além de serem relativamente grandes e pouco praticos de se utilizar dentro do carro. De forma a facilitar o uso da bancada todo o comando do sistema sera realizado através de um aplicativo para celular. Nele sera possivel acompanhar o estado da bancada, controlar o teste, inserir informaçoes para avaliaçao posterior, assim como acompanhar as mediçoes em tempo real, de modo a identificar possiveis problemas de forma rapida.

Sera desenvolvido ainda um software de tratamento e analise de dados com interaface grafica e uso facilitado, este sim a ser utilizado em computador, devido a maior flexibilidade do mesmo para trabalhos mais longos.

\section{Projeto Mecânico}

Para o projeto mecanico foram levantadas as seguintes necessidades:

\begin{itemize}
    \item Facilidade construtiva, de forma a tornar rápida a construção e manutenção da bancada
    \item Leveza, de modo a não se criar uma barreira quanto ao uso da mesma
    \item Baixo arrasto aerodinâmico, a fim de diminuir a influência da estrutura nas medições
    \item Rigidez, para que as forças e momentos medidos sejam realmente os modelados
    \item Baixo custo
\end{itemize}

Uma soluçao que responde de forma positiva a maioria dessas necessidades é a de uma estrutura composta de perfis extrudados de aluminio. Este tipo de estrutura é muito comum de ser encontrada em laboratorios ou fabricas, sendo comumente utilizada para a construçao de bancadas experimentais e estaçoes de trabalho.

\begin{figure}[!ht]
    \centering
    \includegraphics[width=.8\linewidth]{figuras/placeholder.png}
    \caption{FOTO DO PERFIL DE ALUMINIO\cite{autor}.}
    \label{fig:placeholder}
\end{figure}

O contra desta estrutura fica pelo provavel alto arrasto aerodinamico. Este problema porem pode ser contornado posteriormente carenando-se a estrutura, semelhante ao que se faz em aeronaves. Ainda assim este arrasto, com ou sem carenagem, deve ser levado em conta nos testes e a maneira levantada de se mensurar esta grandeza é realizar os testes com a bancada sem geometria acoplada, medindo-se assim o arrasto aerodinamico da propria bancada e descontando este valor das mediçoes posteriores.

De modo a se medir o arrasto, duas soluçoes sao possiveis: 

\begin{enumerate}
    \item Colocar uma celula de carga horizontal e dar liberdade de mvoimento para a bancada no eixo X
    \item Ter como unicas restriçoes no eixo X células de carga
\end{enumerate}

\begin{figure}[!ht]
    \centering
    \includegraphics[width=.8\linewidth]{figuras/placeholder.png}
    \caption{ESQUEMATICO DAS SOLUCOES 1 E 2\cite{autor}.}
    \label{fig:placeholder}
\end{figure}

Devido ao fato de que a segunda soluçao exigiria ao menos tres celulas de carga e as colocaria como componentes estruturais (ja que nao existiria outra ligaçao estrutural entre a parte inferior e superior da bancada) a primeira opçao foi escolhida.

Para dar liberdade de movimentaçao no eixo X um conjunto de guias lineares e fixaçoes rolamentadas foi escolhido, devido a seu baixo atrito e baixo custo. O contra da soluçao com apenas uma celula de carga se da justamente devido a existencia desta força de atrito das guias no eixo X, que a principio existe e nao é medida. Esta força pode porem ser avaliada num primeiro momento, em laboratorio, e compensada posteriormente.
    
\section{Projeto Eletrônico}

Para o sistema eletronico da bancada, tomou-se como ponto de partida a telemetria T2016, desenvolvida pela equipe Céu Azul inicialmente para utilização nos projetos da classe Advanced.

Tal sistema consiste de um computador central, modelo Raspberry Pi 3 B+, rodando um sistema operacional Linux, com sensores conectados como perifericos. Entre os sensores ja disponiveis na equipe estavam:

\begin{itemize}
    \item Tubos de pitot com transdutores MPVX7002DP
    \item GNSS modelo uBlox neo-6m
    \item IMU modelo GY85, com acelerometro de 3 eixos, giroscopio de 3 eixos, magnetometro de 3 eixos, barometro e termometro
\end{itemize}

Alem destes sensores o sistema possuira um par de rádios seriais de 433MHz, que pode ser utilizdo para transmissão dos dados em tempo real para o celular, comunicando a bancada com o aplicativo controlado pelo operador.

Para a mediçao das forças na bancada foi necessaria a aquisiçao de células de carga e transdutores de célula de carga. Devido ao baixo custo optou-se por células sem marca. Esta decisao implica contudo num custo extra de tempo para a caracterizaçao das células.

Para a transduçao dos dados da celula (de resistencia para tensao e posteriormente para força) foram adquiridos módulos HX711. Estes modulos sao bastante comuns no mercado e implementam num mesmo CI a alimentaçao e leitura da ponte de wheatstone, alem da transformaçao em dados digitais. A maior vantagem deste CI é provavelmente a alimentaçao integrada da ponte de Wheatstone, que é no caso ja estabilizada e filtrada, resultando em uma alta relaçao sinal-ruido. Em geral essa alimentaçao acontece em circuitos discretos separados do CI de leitura, e acabam resultando numa pior relaçao sinal-ruido.

\section{Projeto de Software Embarcado}

O software que roda de forma embarcada na plataforma central consiste em uma serie de modulos escritos em Python com funçoes desmembradas. Entre as funçoes necessarias no software destacam-se:

\begin{itemize}
    \item Aquisiçao de dados dos sensores
    \item Parseamento dos dados para padrao comum
    \item Transmissao serial de dados via radio
    \item Recebimento e interpretaçao de comandos externos
    \item Gravaçao dos dados no sistema
    \item Coordenaçao e sincronia dos processos anteriores
\end{itemize}

A figura X mostra a arquitetura do software divida em seus varios modulos.

\begin{figure}[!ht]
    \centering
    \includegraphics[width=.8\linewidth]{figuras/placeholder.png}
    \caption{ESQUEMATICO DO SOFTWARE EMBARCADO\cite{autor}.}
    \label{fig:placeholder}
\end{figure}

Este software ja existia em versao primaria na plataforma T2016 e foi refatorado para uso na nova plataforma, permitindo que suas funçoes fossem extendidas. A totalidade das mudanças compreendidas por este trabalho pode ser vista no repositorio Git do projeto [X]. 

\section{Projeto de Software de Analise}

O software de analise de dados tem por funçao receber os dados "crus" gerados pela bancada e entregar dados uteis processados, com suas devidas incertezas estimadas.

Entre as funçoes desejadas neste software estao:

\begin{itemize}
    \item Apresentar uma interface amigavel para uso facilitado
    \item Receber os dados "crus"
    \item Filtrar cada dado conforme especificaçoes previas ou personalizadas pelo usuario
    \item Apresentar os dados crus e processados na forma de graficos
    \item Permitir interaçao do usuario com os dados
    \item Exportar os dados em formato util, seja na forma textual, em planilhas ou mesmo diretamente como graficos
\end{itemize}

A figura X mostra a arquitetura deste software.

\begin{figure}[!ht]
    \centering
    \includegraphics[width=.8\linewidth]{figuras/placeholder.png}
    \caption{ESQUEMATICO DO SOFTWARE DE ANALISE\cite{autor}.}
    \label{fig:placeholder}
\end{figure}

\section{MVP}

Para validar a ideia desta bancada foi proposto um MVP que possuise as principais caracteristicas da soluçao proposta, mas que pudesse ser construido com materiais ja disponiveis pela equipe.

Do ponto de vista de projeto mecanico, esta primeira versao da bancada teve sua estrutura construida em madeira, utilizou corrediças de gaveta para dar liberdade de movimento no eixo X e usava como torre uma estrutura de aço. A escolha por estas soluçoes se deu pela ja disponibilidade das mesmas na equipe Céu Azul.

Do ponto de vista de projeto eletronico todos os componentes ja estavam presentes, com exceçao do pitot de multiplos angulos, das celulas de carga e dos modulos HX711, sendo estes ultimos dois adquiridos.

Do ponto de vista de software o aplicativo para controle da bancada foi desenvolvido de forma preliminar enquanto o software de tratamento e analise de dados ainda nao existia.

As fotos a seguir mostram o MVP finalizado.

\begin{figure}[!ht]
    \centering
    \includegraphics[width=.8\linewidth]{figuras/placeholder.png}
    \caption{FOTO DA BANCADA 1.0 - 1\cite{autor}.}
    \label{fig:placeholder}
\end{figure}

\begin{figure}[!ht]
    \centering
    \includegraphics[width=.8\linewidth]{figuras/placeholder.png}
    \caption{FOTO DA BANCADA 1.0 - 2\cite{autor}.}
    \label{fig:placeholder}
\end{figure}

Foi realizada uma serie de testes com esta bancada, entre eles o de mediçao de forças aerodinamicas em uma asa, de mediçao do empuxo dinamico em motor e de mediçao do momento causado pelo acionamento de superficies de comando (ailerons) em uma asa. O procedimento destes testes e seus resultados esta detalhado em [artigo_bancada_1]. Destacam-se aqui as principais conclusoes levantadas por [artigo_bancada_1]:

\begin{itemize}
    \item ponto1
    \item ponto2
    \item ponto3
\end{itemize}

Fica claro portanto que o MVP foi um sucesso no sentido de que provou o funcionamento da bancada proposta, porém possuia problemas estruturais intrinsecos as escolhas para a prototipagem mecanica, e portanto nao demonstrou fidelidade suficiente para seu uso. Além disso, ficaram claros diversos problemas do ponto de vista de praticidade, entre eles:

\begin{itemize}
    \item Aplicativo ainda em estagio inicial, apresentando uma interface pouco intuitiva
    \item Dificuldade para se ajustar o angulo de incidencia do dispositivo testado
    \item Problemas recorrentes de mal-contato elétrico, resultando na perda de baterias inteiras de dados
    \item Pouco controle sobre o estado e funcionamento do computador da bancada
    \item Configuraçao do teste no computador da bancada era realizado modificando-se o script original, o que tornava esta configuraçao bastante suscetivel a erros pelo operador 
    \item Software embarcado (no computador da bancada) pouco organizado, com as estruturas de funcionamento do software altamente entrelaçadas
    \item Falta de sincronia entre os dados das celulas de carga com relaçao aos outros sensores, o que tornava o processamento dos dados um processo bastante massante
    \item Falta de informaçoes sobre o teste no arquivo de gravaçao dos dados, o que fazia com que o operador precisasse recorrer a outros meios para guardar a informaçao sobre o que e como estava sendo feito em cada bateria de testes, causando muitas vezes a perda de informaçao sobre as mesmas
    \item Falta de software para analise dos dados, o que tornava o processo de utilizaçao dos dados gerados pela bancada uma tarefa altamente especializada
\end{itemize}

Estes pontos foram tomados como base para a segunda versao da bancada, detalhada neste texto.

\section{Primeira versao final}

Levando em conta os pontos levantados no MVP deu-se inicio a modelagem e construçao da primeira versao final da bancada.

\subsection{Modelo físico}

COMENTAR SOBRE O MODELO ESTATICO E A CORRECAO DE FALSA SUSTENTACAO

\subsection{Mecanica}

A estrutura da bancada utilizando perfis extrudados em aluminio foi modelada no software Solidworks.

\begin{figure}[!ht]
    \centering
    \includegraphics[width=.8\linewidth]{figuras/placeholder.png}
    \caption{RENDERINGS DA ESTRUTURA DA BANCADA\cite{autor}.}
    \label{fig:placeholder}
\end{figure}

Os perfis de aluminio sao adquiridos em barras de 1 metro de comprimento. A modelagem permitiu assim a definiçao das medidas dos cortes a serem realizados e a otimizaçao do uso das barras.

Utilizando a estimativa de esforços assumida previamente foi desenvolvida a estrutura da bancada realizando-se simulaçoes estruturais no software Ansys Mechanical de modo a se avaliar os deslocamentos maximos da estrutura. Para o projeto decidiu-se por nao permitir deslocamentos verticais maiores que Xmm e deslocamentos horizontais maiores que Xmm. Esta restriçao visa impedir que a deformaçao da bancada induza erros de alinhamento e consequentemente misture as medidas de sustentaçao e arrasto [X].

\begin{figure}[!ht]
    \centering
    \includegraphics[width=.8\linewidth]{figuras/placeholder.png}
    \caption{RENDERING DA SIMULACAO DE ESFORÇOS MECANICOS NA BANCADA\cite{autor}.}
    \label{fig:placeholder}
\end{figure}

As conexoes estuturais foram feitas com cantoneiras planas de aço, garantindo rigidez nas juntas, o que agrega fidelidade na simulaçao (uma vez que nela as juntas sao consideradas rigidas).

Para o encaixe das células de carga na bancada, assim como dos suportes das guias lineares e dos pillow-blocks, foram projetados adaptadores que posteriormente foram produzidos via manufatura aditiva de polímero. Este método de manufatura permitiu a rapida prototipagem dessas peças e acelerou o desenvolvimento do projeto.

\begin{figure}[!ht]
    \centering
    \includegraphics[width=.8\linewidth]{figuras/placeholder.png}
    \caption{FOTO DAS PEÇAS IMPRESSAS - 1\cite{autor}.}
    \label{fig:placeholder}
\end{figure}

\begin{figure}[!ht]
    \centering
    \includegraphics[width=.8\linewidth]{figuras/placeholder.png}
    \caption{FOTO DAS PEÇAS IMPRESSAS - 2\cite{autor}.}
    \label{fig:placeholder}
\end{figure}

\begin{figure}[!ht]
    \centering
    \includegraphics[width=.8\linewidth]{figuras/placeholder.png}
    \caption{FOTO DAS PEÇAS IMPRESSAS - 3\cite{autor}.}
    \label{fig:placeholder}
\end{figure}

Devido a dificuldade na simulaçao de peças produzidas por este processo, que possuem alta anisotropia e grande dispersao nos resultados dependendo da qualidade da impressao, foram realizados testes estruturais estaticos para se garantir que as peças nao falhassem. 

Cabe aqui um adendo de que, para se garantir ainda maior rigidez a bancada como um todo, é de interesse a produçao dessas peças em metal, realizando as devidas modificaçoes para melhor se adequar ao processo de manufatura escolhido.

COMENTAR SOBRE A GUIA LINEAR E SEU MENOR ATRITO E MELHOR ALINHAMENTO

COMENTAR SOBRE OS CABOS DE ACO

COMENTAR SOBRE O AJUSTE DE ANGULO DE INCIDENCIA

\subsection{Eletronica}

Como um dos maiores problemas do MVP foi o de falha no sistema eletronico devido a mal-contato, atençao especial foi dada a esta parte. Duas soluçoes foram levantadas:

\begin{itemize}
    \item Produçao de placas de circuito impresso para o sistema
    \item Utilizaçao de conectores mais robustos
\end{itemize}

Foram produzidas duas PCBs: uma para os transdutores de celula de carga e o Arduino responsavel pela aquisiçao de seus sinais e outra para os transdutores de pressao e o ADC responsavel pela conversao do seu sinal.

\begin{figure}[!ht]
    \centering
    \includegraphics[width=.8\linewidth]{figuras/placeholder.png}
    \caption{FOTO DA PCB - 1\cite{autor}.}
    \label{fig:placeholder}
\end{figure}

\begin{figure}[!ht]
    \centering
    \includegraphics[width=.8\linewidth]{figuras/placeholder.png}
    \caption{FOTO DA PCB - 2\cite{autor}.}
    \label{fig:placeholder}
\end{figure}

Para ambas as placas foram confeccionados cases para alocaçao das placas e dos conectores. Os conectores usados neste projeto foram do modelo GX16, popularmente conhecidos como "conectores de aviaçao". Estes possuem trava rosqueada e conexao bastante firme, garantindo o contato eletrico. Estes conectores ficam presos no case e nao na placa, de modo que em caso de acidentes de tracionamento dos conectores, por exemplo, a PCB fica intacta.

\begin{figure}[!ht]
    \centering
    \includegraphics[width=.8\linewidth]{figuras/placeholder.png}
    \caption{FOTO DOS CASES+PCB+CONECTORES\cite{autor}.}
    \label{fig:placeholder}
\end{figure}

\subsection{Software Embarcado}

Como ja mencionado, todo o software embarcado foi refatorado da versao MVP para esta primeira versao final. Alguns pontos principais merecem destaque:

\begin{itemize}
    \item Integraçao do Arduino como sensor na plataforma central
    
    \item Implementaçao de nova rotina de ligaçao da bancada 
\end{itemize}

Sobre o primeiro ponto, na versao MVP o Arduino que fazia a aquisiçao do sinal das celulas nao se comunicava com o Raspberry Pi (plataforma central). Deste modo, seus dados nao possuiam sincronia, e esta deveria ser feita posteriormente por quem fosse se utilizar dos dados. Alem de exigir um grande trabalho manual, este processo muitas vezes nao ficava com qualidade satisfatoria, e os dados acabavam desconectados. Para resolver este problema o Arduino passou a ser conectado na plataforma central e tratado por esta como um sensor, que respondia a pedidos de envio de dados. Para garantir robustez na comunicaçao foi implementado comunicaçao de dois sentidos com confirmaçao de recebimento e checksum.  

Quanto ao segundo ponto, a nova rotina garante que existe comunicaçao sempre que um novo comando é acionado, e implementa a passagem de configuraçoes por parte do operador para a plataforma. Este ultimo ponto foi essencial para que detalhes do teste, como o dispositivo testado, seu angulo de incidencia e outras condiçoes do teste pudessem ser registradas no arquivo de dados, facilitando o entendimento de quem o fosse processar e interpretar.

\subsection{Software de Analise}

DESCREVER SOFTWARE DE ANALISE

\subsection{Sensoriamento}

\begin{itemize}
    \item Sonda de angulo de ataque
    \item IMU
\end{itemize}

Foram posicionados 2 tubos de pitot na bancada. A disposição dos tubos se deu de forma a procurar avaliar a velocidade dois pontos: fluxo livre (longe de distorções causadas pelo conjunto carro/bancada/asa) e logo antes da asa (procurando identificar alguma diferença entre a velocidade de fluxo livre e a sentida pela asa).

\begin{figure}[!ht]
    \centering
    \includegraphics[width=.8\linewidth]{figuras/placeholder.png}
    \caption{ESQUEMATICO DA DISPOSICAO DOS TUBOS DE PITOT\cite{autor}.}
    \label{fig:placeholder}
\end{figure}

\subsection{Bancada final}

As figuras X e X mostram a modelagem final da bancada e sua versao real construida.

\begin{figure}[!ht]
    \centering
    \includegraphics[width=.8\linewidth]{figuras/placeholder.png}
    \caption{RENDERING FINAL DA BANCADA\cite{autor}.}
    \label{fig:placeholder}
\end{figure}

\begin{figure}[!ht]
    \centering
    \includegraphics[width=.8\linewidth]{figuras/placeholder.png}
    \caption{FOTO DA BANCADA FINAL CONSTRUIDA\cite{autor}.}
    \label{fig:placeholder}
\end{figure}

\section{Testes}

Uma serie de testes foi feita inicialmente de forma mais rapida visando-se buscar melhorias a serem feitas na bancada, seja na parte mecanica, eletronica ou software. Estes testes se deram simultaneamente a fase final de projeto.

Uma vez finalizada a versao final da bancada, foi iniciada a conduçao dos testes comparativos.

\subsection{O dispositivo de teste}

Os testes comparativos tem por finalidade gerar dados sobre geometrias conhecidas que possam ser confrontados com literatura existente. Para este a seguinte geometria foi utilizada:

\begin{itemize}
    \item Asa retangular
    \item Perfil Selig1223
    \item Envergadura de 2m
    \item Corda de 30cm
\end{itemize}

A faixa de Reynolds utilizada nos testes foi de X a Y, correspondendo a velocidades de X a Y, compativeis com as aeronaves desenvolvidas para a competiçao SAE Aerodesign.

Esta asa foi construida em isopor laminado com fibra de vidro e testes com a mesma foram realizados, porém devido a uma torçao do perfil durante a laminaçao da mesma, os resultados se mostraram bastante dispares com a teoria. Por este motivo se optou por utilizar nos testes a asa do prototipo 2019 da equipe, que se encontra detalhada na modelagem a seguir:

\begin{figure}[!ht]
    \centering
    \includegraphics[width=.8\linewidth]{figuras/placeholder.png}
    \caption{MODELAGEM COM DIMENSOES DA ASA 2019.\cite{autor}.}
    \label{fig:placeholder}
\end{figure}

A faixa de Reynolds utilizada nos testes foi de X a Y, correspondendo a velocidades de X a Y.

Para uma primeira estimativa do desempenho de tal asa se utilizou a teoria X.

CALCULOS DA TEORIA X

Em um segundo momento utilizou-se o software de codigo-aberto AVL, que se utiliza de um modelo VLM para a simulaçao da asa.

\begin{figure}[!ht]
    \centering
    \includegraphics[width=.8\linewidth]{figuras/placeholder.png}
    \caption{RENDERING DA SIMULACAO NO AVL\cite{autor}.}
    \label{fig:placeholder}
\end{figure}

\begin{figure}[!ht]
    \centering
    \includegraphics[width=.8\linewidth]{figuras/placeholder.png}
    \caption{FOTO DA ASA CONSTRUIDA\cite{autor}.}
    \label{fig:placeholder}
\end{figure}

A asa foi construida segundo o metodo tradicional da equipe, com perfis cortados em MDF 6mm, reforços estruturais em estrutura-sanduiche com espuma de PVC e fibra de carbono e carenagem de fita adesiva. Esta asa corresponde ao 1o prototipo do ano de 2019.

Simulaçoes estruturais para tal asa foram realizadas buscando-se avaliar a torçao geometrica e a deflexao desenvolvidas durante sua operaçao. Uma torçao maxima de X graus e um deslocamento de ponta maximo de X mm foram encontrados. Segundo X, a perda de sustentaçao para tal condiçao seria de X, valor baixo e portanto ignorado durante a avaliaçao dos resultados. 

\subsection{O teste}

Os testes foram conduzidos em uma avenida reta e comprida (Avenida Beira-mar Norte), excencialmente plana e em horário de trânsito reduzido (durante a madrugada).

\begin{figure}[!ht]
    \centering
    \includegraphics[width=.8\linewidth]{figuras/placeholder.png}
    \caption{PRINT DO MAPS MOSTRANDO CIRCUITO DE TESTE.\cite{autor}.}
    \label{fig:placeholder}
\end{figure}

O teste foi conduzido acelerando-se o carro até uma velocidade de aproximadamente 15m/s (54km/h) e mantendo essa velocidade durante o maior tempo possivel. Sendo conduzido desta maneira o teste entrega resultados para apenas um valor de Reynolds, e nao em uma faixa de Reynolds como foi inicialmente estipulado. Foi assim feito pois nos testes preliminares verificou-se que cada bateria de testes demorava um tempo consideravel, e repetir ela para uma grande faixa de velocidades reduziria o numero de dados para cada uma dessas velocidades. Decidiu-se portanto ter um maior numero de dados para um mesmo Reynolds, garantindo-se uma qualidade melhor dos resultados finais. Com mais tempo porem, ou uma conduçao mais rapida dos testes, poder-se-ia (sdds presidento) realizar os testes para mais valores de Reynolds.  

\begin{figure}[!ht]
    \centering
    \includegraphics[width=.8\linewidth]{figuras/placeholder.png}
    \caption{FOTO DO APP SENDO USADO\cite{autor}.}
    \label{fig:placeholder}
\end{figure}

Para se averiguar a velocidade foram utilizados tres recursos: velocimetro do proprio carro, velocidade medida pelo modulo GNSS e velocidade medida pelo tubo de pitot posicionado em fluxo livre. Idealmente as tres medidas devem apresentar resultado semelhante. Para se acompanhar a velocidade medida pelos sensores foi utilizado o aplicativo desenvolvido.

\begin{figure}[!ht]
    \centering
    \includegraphics[width=.8\linewidth]{figuras/placeholder.png}
    \caption{FOTO DO TESTE - 1\cite{autor}.}
    \label{fig:placeholder}
\end{figure}

\begin{figure}[!ht]
    \centering
    \includegraphics[width=.8\linewidth]{figuras/placeholder.png}
    \caption{FOTO DO TESTE - 2\cite{autor}.}
    \label{fig:placeholder}
\end{figure}

\begin{figure}[!ht]
    \centering
    \includegraphics[width=.8\linewidth]{figuras/placeholder.png}
    \caption{FOTO DO TESTE - 3\cite{autor}.}
    \label{fig:placeholder}
\end{figure}

Para cada bateria de testes a asa foi instalada em um ângulo de incidencia diferente. Foram realizados testes para os ângulos de 0, 3, 6, 9, 12 e 15 graus. Para cada um dos angulos, tres baterias de teste foram conduzidas, buscando-se criar uma base de dados grande e diminuir a influencia dos testes no resultado.

