\chapter{Conclusão}\label{chp:conc}

    Este trabalho procurou descrever o projeto, construção e teste de uma bancada experimental para caracterização aerodinâmica de componentes de VANTs, visando eliminar a necessidade de testes em túnel de vento ou ensaios em voo.
    
    O trabalho foi desenvolvida no âmbito de uma equipe de projeto aeronáutico, buscando preencher a necessidade da mesma por dados experimentais que norteassem a validação e otimização de seus projetos.
    
    O trabalho foi bem sucedido no sentido de projetar, construir e testar tal bancada, cumprindo os requisitos estabelecidos de custo, facilidade de uso, desenvolvimento de softwares de suporte e documentação. 
    
    Os resultados obtidos no teste estão dentro do esperado quando comparados as simulações para os mesmos componentes testados.  Não foram testados contudo componentes com valores experimentais conhecidos, de forma que não se pode dizer que a bancada esta validada. Esperava-se ao inicio deste trabalho validar ou invalidar a bancada, porém a necessidade de finalizar os ensaios experimentais antes do previsto impediu que tal ação fosse concluída.
    
    Durante o desenvolvimento deste trabalho, dificuldades surgiram onde se poderia e onde não se poderia imaginar. Atrasos na entrega dos componentes, cargas não previstas, erros na execução dos testes, comportamentos inesperados do software de aquisição de dados, comportamentos inesperados dos sensores, todos foram problemas encontrados durante o caminho, felizmente resolvidos, um a um, ainda que atrasando o cronograma previsto.
    
    O uso da bancada hoje se encontra bastante facilitado, sendo possível realizar os testes e a analise dos resultados em uma noite de trabalho, considerando um bom planejamento e execução dos mesmos.
    
\section{Trabalhos futuros}
    
    Pode-se dizer que este trabalho se encontra em estagio avançado, mas ainda não concluído. A principal necessidade que se enxerga para um trabalho futuro é a validação da mesma para componentes já experimentalmente caracterizados. Esta validação agregaria em muito e permitiria que a equipe Céu Azul passasse a realizar seus ensaios experimentais visando otimizar o projeto com uma maior garantia sobre a validade dos resultados. Além da própria validação, alguns pontos a se tratar no futuro, seja por este autor ou por um terceiro, são levantados:
    
\begin{itemize}
    \item Mudança na solução da bandeja inferior da balança, passando a utilizar 3 células de carga e eliminando as guias lineares.
    \item Calibração dos fatores de acoplamento entre medições de sustentação e arrasto
    \item Validação da bancada contra geometrias com valores experimentais já consolidados, tais como esferas, cilindros e asas de perfil NACA
    \item Desenvolvimento de interface gráfica para carregamento dos dados de testes, visando eliminar a necessidade de execução das analises via scripts.
    \item Inclusão de rotina de calibração das células de carga e dos tubos de Pitot via app, tornando a calibração um processo rápido e facilitado.
    \item Inclusão de rotina de testes de forma guiada pelo app, isto é, em um formato de passo-a-passo, com instruções de execução, avisos de mal-uso ou resultados inconsistentes e automatização das analises.
    \item Possibilidade de transferência dos dados da plataforma de aquisição para o celular via Bluetooth, eliminando a necessidade de retirada do cartão de memoria da mesma.
    \item Inclusão de modo de gravação de áudio no app, visando a possibilidade de uma "narração" do teste pelo responsável. Este áudio permitiria que avaliações extras pudessem ser realizadas pelo responsável pela analise, como identificação de possíveis erros de execução, anormalidades ou mesmo testes diferenciados, como a execução de comandos na aeronave a serem posteriormente caracterizados. 
\end{itemize}
