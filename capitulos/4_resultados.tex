\chapter{Resultados}\label{chp:res}

O conjunto de dados, originais e derivados (aqueles criados a partir de dados originais), estão apresentados na tabela X.

\begin{table}[!ht]
    \centering
    \includegraphics[width=.8\linewidth]{figuras/outras/placeholder.png}
    \caption{TABELA COM DADOS GRAVADOS\cite{autor}.}
    \label{fig:placeholder}
\end{table}

Na imagem X pode ser visto um exemplo dos dados adquiridos durante uma bateria de testes. A gravação dos dados na plataforma ocorre a uma taxa fixa de X Hz, diferente da taxa original de aquisição de cada sensor. 

\begin{table}[!ht]
    \centering
    \includegraphics[width=.8\linewidth]{figuras/outras/placeholder.png}
    \caption{GRAFICOS COM FREQUENCIA DE AQUISICAO DE CADA SENSOR (SINAL DIGITAL)\cite{autor}.}
    \label{fig:placeholder}
\end{table}

Isso é feito de modo que todos apresentem uma mesma frequência de ocorrência na plataforma e portanto tenham sempre correspondência nos dados dos outros sensores, facilitando posterior processamento. Neste nivelamento alguns dados, por possuírem frequência de aquisição mais baixa que a de gravação da plataforma, acabam se repetindo até que novos valores estejam disponíveis.

\begin{figure}[!ht]
    \centering
    \includegraphics[width=.8\linewidth]{figuras/outras/placeholder.png}
    \caption{PLOTS OCUPANDO A PAGINA INTEIRA - 1\cite{autor}.}
    \label{fig:placeholder}
\end{figure}

\begin{figure}[!ht]
    \centering
    \includegraphics[width=.8\linewidth]{figuras/outras/placeholder.png}
    \caption{PLOTS OCUPANDO A PAGINA INTEIRA - 2\cite{autor}.}
    \label{fig:placeholder}
\end{figure}

\begin{figure}[!ht]
    \centering
    \includegraphics[width=.8\linewidth]{figuras/outras/placeholder.png}
    \caption{PLOTS OCUPANDO A PAGINA INTEIRA - 3\cite{autor}.}
    \label{fig:placeholder}
\end{figure}

\section{Limpeza, filtragem e redução dos dados}

\subsection{Limpeza de dados}

A premissa principal para o uso dos dados do teste é o de que os valores medidos estejam dentro do intervalo de uso dos sensores e de que as acelerações sobre o sistema sejam baixas. O intervalo de uso dos sensores se encontra na tabela X.

\begin{table}[!ht]
    \centering
    \includegraphics[width=.8\linewidth]{figuras/outras/placeholder.png}
    \caption{TABELA COM ZONA DE USO DE CADA SENSOR\cite{autor}.}
    \label{fig:placeholder}
\end{table}

O principal sensor a limitar o uso dos dados foi o Pitot. Abaixo de X m/s o sensor é incapaz de medir a velocidade corretamente, portanto todos os dados adquiridos abaixo dessa velocidade foram descartados, como pode ser visto em exemplo na figura X.

\begin{figure}[!ht]
    \centering
    \includegraphics[width=.8\linewidth]{figuras/outras/placeholder.png}
    \caption{PLOTS DE VELOCIDADE SUSTENTACAO E ACELERACAO Z\cite{autor}.}
    \label{fig:placeholder}
\end{figure}

Os dados resultantes desse processo são considerados validos, ainda que possam não ser interessantes para a medição por não se encaixarem no modelo proposto (i.e se forem medidos sob altas acelerações). Dados com aceleração em X, Y ou Z maior que 0.5g foram descartados no próximo sub-set.

Ainda que os dados resultantes desses dois processos sejam validos e utilizáveis no modelo, os mesmos podem não se encontrar na faixa de Reynolds estipulada para o teste. Dado que o teste foi executado a uma velocidade praticamente constante, o resultado final sera dado para apenas um valor de Reynolds, sendo assim, dados adquiridos que sejam validos mas não estejam próximos ao Reynolds do teste não devem ser utilizados, dado que devem apresentar comportamento aerodinâmico diferente daquele correspondente ao Reynolds desejado[X]. Desta forma os dados finais a serem utilizados para medição serão aqueles correspondentes aos instantes onde foi alcançada velocidade praticamente constante e com um máximo de 10\% de desvio com relação ao Reynolds desejado estipulado.

\begin{figure}[!ht]
    \centering
    \includegraphics[width=.8\linewidth]{figuras/outras/placeholder.png}
    \caption{PLOTS COM PATAMARES DE VELOCIDADE CONSTANTE\cite{autor}.}
    \label{fig:placeholder}
\end{figure}

\subsection{Filtragem}

Apos a limpeza dos dados, onde foram removidos aqueles que estavam fora da zona de interesse, buscou-se avaliar a filtragem do sinal, visando aumentar a relação sinal-ruido. Para isto foi realizada uma analise FFT[X] nos dados de carga, velocidade e aceleração.

\begin{figure}[!ht]
    \centering
    \includegraphics[width=.8\linewidth]{figuras/outras/placeholder.png}
    \caption{ANALISES FFT\cite{autor}.}
    \label{fig:placeholder}
\end{figure}

A partir do resultado das analises foram dimensionados filtros do tipo passa-baixa para a remoção dos ruídos de alta frequência em cada um dos dados[X]. Os resultados se encontram na figura X.

\begin{figure}[!ht]
    \centering
    \includegraphics[width=.8\linewidth]{figuras/outras/placeholder.png}
    \caption{DADOS APOS PASSA-BAIXA\cite{autor}.}
    \label{fig:placeholder}
\end{figure}

\subsection{Redução}

Em posse dos dados filtrados passa-se a etapa de redução. Não é do interesse do projetista a principio saber qual sustentação, arrasto ou momento foram alcançados durante os testes, pois esses são dados absolutos que dependem do tamanho do componente e da velocidade no instante medido[X]. Para facilidade de analise e mais justa comparação entre diferentes componentes é de interesse a redução destes dados na forma dos coeficientes aerodinâmicos[X].

EQUACAO DOS COEFICIENTES AERODINÂMICOS

As equações XYZ foram aplicadas a todas as amostras de dados gerando três novas colunas: CL, CD e CM. Para cada bateria de teste foi calculado o valor médio de cada coeficiente, e em seguida foi calculado para cada angulo de incidência o valor médio dos coeficientes das baterias, assim como seu desvio padrão.

Os valores finais de media e desvio padrão para os coeficientes de cada angulo de incidência estão explicitados nos gráficos X Y e Z.

\begin{figure}[!ht]
    \centering
    \includegraphics[width=.8\linewidth]{figuras/outras/placeholder.png}
    \caption{GRAFICOS CX VS ALPHA COM DESVIO PADRAO\cite{autor}.}
    \label{fig:placeholder}
\end{figure}

DISCUTIR RESULTADOS E AVALIAR A INCERTEZA DA BANCADA

