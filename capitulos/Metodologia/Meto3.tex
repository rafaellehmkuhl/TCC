\subsection{Eletronica}

Como um dos maiores problemas do MVP foi o de falha no sistema eletrônico devido a mal-contato, atenção especial foi dada a esta parte. Duas soluções foram propostas:

\begin{itemize}
    \item Produção de placas de circuito impresso para todos os sistemas
    \item Utilização de conectores mais robustos
\end{itemize}

Foram projetadas e produzidas duas PCBs: uma para os transdutores de célula de carga e outra para os transdutores de pressão diferencial.

\begin{figure}[!ht]
    \centering
    \includegraphics[width=.8\linewidth]{figuras/renders/pitot_board.png}
    \caption{Projeto da placa de transdutores de pressão diferencial no software Eagle\cite{autor}.}
    \label{fig:projeto_pcb_pitot}
\end{figure}

\begin{figure}[!ht]
    \centering
    \includegraphics[width=.8\linewidth]{figuras/construcao/pcbs_montadas_2.jpg}
    \caption{Placa de transdutores de pressão diferencial fresada e com sensores instalados\cite{autor}.}
    \label{fig:placa_pcb_pitot}
\end{figure}

\begin{figure}[!ht]
    \centering
    \includegraphics[width=.8\linewidth]{figuras/renders/loadcell_board.png}
    \caption{Projeto da placa de transdutores ce célula de carga no software Eagle\cite{autor}.}
    \label{fig:projeto_pcb_celulas}
\end{figure}

\begin{figure}[!ht]
    \centering
    \includegraphics[width=.8\linewidth]{figuras/construcao/pcbs_montadas_3.jpg}
    \caption{Placa de transdutores de célula de carga fresada e com sensores instalados\cite{autor}.}
    \label{fig:placa_pcb_celulas}
\end{figure}

Foram confeccionados cases para alocação de ambas as placas e seus conectores. Os conectores usados neste projeto foram do modelo GX16, popularmente conhecidos como "conectores de aviação". Estes possuem trava rosqueada e ótimo contato elétrico. Os conectores ficam presos no case e não na placa, de modo que em caso de acidentes de tracionamento dos cabos as placas não sejam danificadas.

\begin{figure}[!ht]
    \centering
    \includegraphics[width=.8\linewidth]{figuras/outras/placeholder.png}
    \caption{FOTO DOS CASES+PCB+CONECTORES\cite{autor}.}
    \label{fig:placeholder}
\end{figure}

\begin{figure}[!ht]
    \centering
    \includegraphics[width=.8\linewidth]{figuras/outras/placeholder.png}
    \caption{FOTO DA CONEXAO INTERNA COM FIOS\cite{autor}.}
    \label{fig:placeholder}
\end{figure}

\subsection{Software Embarcado}

Como já mencionado, todo o software embarcado foi refatorado da versão MVP para esta primeira versão final. Alguns pontos principais merecem destaque:

\begin{itemize}
    \item Integração do Arduino como sensor na plataforma central, permitindo que mais Arduinos sejam conectados paralelamente de forma simplificada
    \item Implementação de nova rotina de ligação da bancada, adaptando automaticamente as configurações internas aos sensores que estiverem conectados, evitando a necessidade de alterar o código toda vez que um novo teste for executado
    \item Implementação de rotina de configuração do teste remotamente via app, permitindo que o operador inclua no arquivo de dados detalhes do teste como o dispositivo testado, seu angulo de incidência e outras condições, facilitando o entendimento dos dados pelo responsável por processa-los 
    \item Inclusão de código para uso da sonda de angulo de ataque através dos sensores de pressão diferencial
    \item Transferência de todas as configurações do software para um arquivo de configuração simplificada, permitindo que pessoas sem experiencia com o software embarcado consigam realizar configurações avançadas de forma simples
\end{itemize}

Sobre o primeiro ponto, na versão MVP o Arduino que fazia a aquisição do sinal das células não se comunicava com o Raspberry Pi (plataforma central). Deste modo, seus dados não possuíam sincronia, e esta deveria ser feita posteriormente por quem fosse se utilizar dos dados. Além de exigir um grande trabalho manual, este processo muitas vezes não ficava com qualidade satisfatória, e os dados acabavam não sincronizados. Para resolver este problema o Arduino passou a ser conectado na plataforma central e tratado por esta como um sensor, que responde a pedidos de envio de dados. Para garantir robustez na troca de dados foi implementado protocolo de comunicação de dois sentidos com confirmação de recebimento e checksum.  

\subsection{Software de Analise}

DESCREVER SOFTWARE DE ANALISE

\subsection{Sensoriamento}

\begin{itemize}
    \item Sonda de angulo de ataque
    \item IMU
\end{itemize}

\begin{figure}[!ht]
    \centering
    \includegraphics[width=.8\linewidth]{figuras/outras/placeholder.png}
    \caption{SIMULACAO CFD\cite{autor}.}
    \label{fig:placeholder}
\end{figure}

Foram posicionados na bancada um tubo de Pitot e uma sonda de angulo de ataque. O tubo foi posicionado na bancada de forma a medir velocidade de fluxo livre (longe de distorções causadas pelo conjunto carro/bancada/componente). Para auxiliar neste posicionamento uma simulação qualitativa foi realizada em CFD utilizando o código Fluent, disponível na plataforma Ansys. A modelagem foi simplificada em pontos considerados não criticos de modo a evitar problemas de malha e acelerar a convergência dos resultados.

\begin{figure}[!ht]
    \centering
    \includegraphics[width=.8\linewidth]{figuras/outras/placeholder.png}
    \caption{ESQUEMATICO DA DISPOSICAO DOS TUBOS DE PITOT\cite{autor}.}
    \label{fig:placeholder}
\end{figure}

A sonda de angulo de ataque deve ser fixada ao componente a ser testado, com alinhamento junto a linha de referencia horizontal do mesmo (i.e. linha de corda da raiz da asa). Este posicionamento permite que o angulo de ataque real seja medido pela sonda. 

Para o protótipo a ser testado foi confeccionado um suporte para a sonda, a ser instalado no bordo de ataque da asa. Suportes como este podem ser modelados e impressos rapidamente, e permitem um melhor alinhamento da sonda ao componente.

\begin{figure}[!ht]
    \centering
    \includegraphics[width=.8\linewidth]{figuras/renders/sonda_aoa_asa_superior.png}
    \caption{Detalhe do Pitot de multiplas tomadas (sonda de angulo de ataque) montada na asa superior da aeronave testada\cite{autor}.}
    \label{fig:placeholder}
\end{figure}

\subsection{Bancada final}

As figuras X e X mostram a modelagem final da bancada e mesma construída.

\begin{figure}[!ht]
    \centering
    \includegraphics[width=.8\linewidth]{figuras/outras/placeholder.png}
    \caption{RENDERING FINAL DA BANCADA\cite{autor}.}
    \label{fig:placeholder}
\end{figure}

\begin{figure}[!ht]
    \centering
    \includegraphics[width=.8\linewidth]{figuras/outras/placeholder.png}
    \caption{FOTO DA BANCADA FINAL CONSTRUIDA\cite{autor}.}
    \label{fig:placeholder}
\end{figure}



\section{Testes}

Uma serie de testes foi realizada inicialmente de forma mais rápida visando-se buscar melhorias a serem feitas na bancada, seja na parte mecânica, eletrônica ou de software. Estes testes se deram simultaneamente ao projeto e foram responsáveis por melhorias como:

\begin{enumerate}
    \item Verificação da necessidade de simulação da bancada para otimização da posição do tubo de Pitot.
    \item Verificação da calibração das células de carga e tubos de Pitot.
    \item Inclusão da possibilidade de zeragem das células e Pitots ao inicio do teste via app
    \item Inclusão da IMU para medição das acelerações sofridas durante o teste.
    \item Verificação da necessidade de uso da sonda acoplada ao componente a ser testado
\end{enumerate}

Uma vez finalizada a versão final da bancada, foi iniciada a condução dos testes comparativos.

\subsection{O dispositivo de teste}

Os testes comparativos tem por finalidade gerar dados sobre geometrias conhecidas que possam ser confrontados com literatura existente. Para este teste o dispositivo ideal seria uma asa utilizando perfis do tipo NACA, que possuem vasta literatura de dados experimentais, porém devido a problemas orçamentários e de tempo não foi possível a construção de tal dispositivo, sendo o teste realizado com a asa do primeiro protótipo da equipe para o ano de 2019. A geometria desta asa se encontra a seguir:

\begin{table}[]
\centering
\begin{tabular}{lll}
 & 1a Seçao & 2a Seçao \\
Perfil & Selig1223 & Selig1223 \\
Envergadura & X & X \\
Corda na raiz & X & X \\
Corda na ponta & X & X
\end{tabular}
\end{table}

Foi escolhida a velocidade de X m/s para os testes, visto que é a velocidade de cruzeiro da aeronave. O Reynolds referente a Corda Média Aerodinâmica da asa é de X para as condições do teste.

\begin{figure}[!ht]
    \centering
    \includegraphics[width=.8\linewidth]{figuras/outras/placeholder.png}
    \caption{MODELAGEM COM DIMENSOES DA ASA 2019.\cite{autor}.}
    \label{fig:placeholder}
\end{figure}

Para a simulação de tal asa foi utilizado o software de código-aberto AVL, que implementa o algoritmo Vortex Lattice Method.

\begin{figure}[!ht]
    \centering
    \includegraphics[width=.8\linewidth]{figuras/outras/placeholder.png}
    \caption{RENDERING DA SIMULACAO NO AVL\cite{autor}.}
    \label{fig:placeholder}
\end{figure}

A asa foi confeccionada com nervuras em MDF 6mm, detalhes em madeira balsa, com laminaçao de fibra de carbono nas areas com maiores solicitaçoes, conceito estrutural de caixa de torçao e entelamento com fita Adelbras. Simulações estruturais para tal asa foram realizadas buscando-se avaliar a torçao geométrica e a deflexão desenvolvidas durante sua operação. Uma torçao máxima de X graus e um deslocamento de ponta máximo de X mm foram encontrados. Segundo X, a perda de sustentação para tal condição seria de X, valor baixo e portanto ignorado durante a avaliação dos resultados.

\begin{figure}[!ht]
    \centering
    \includegraphics[width=.8\linewidth]{figuras/outras/placeholder.png}
    \caption{FOTO DA ASA CONSTRUIDA\cite{autor}.}
    \label{fig:placeholder}
\end{figure}

\subsection{O teste}

Os testes foram conduzidos em uma avenida reta e comprida (Avenida Beira-mar Norte), excencialmente plana e em horário de trânsito reduzido (durante a madrugada).

\begin{figure}[!ht]
    \centering
    \includegraphics[width=.8\linewidth]{figuras/internet/here_open_street_maps.png}
    \caption{Local onde foram realizados os testes\cite{autor}.}
    \label{fig:mapa_teste}
\end{figure}

O teste foi conduzido acelerando-se o carro até uma velocidade de aproximadamente 15m/s (54km/h) e mantendo essa velocidade durante o maior tempo possível. Sendo conduzido desta maneira o teste entrega resultados para apenas um valor de Reynolds. Foi assim feito pois em testes preliminares verificou-se que cada bateria de testes demorava um tempo considerável, e repetir ela para uma grande faixa de Reynolds reduziria o numero de dados para cada velocidade. Decidiu-se portanto ter um maior numero de dados para um mesmo Reynolds, garantindo-se uma qualidade melhor dos resultados finais. Com mais tempo porem, ou uma condução mais rápida dos testes, poder-se-ia (sdds presidento) realizar os testes para mais valores de Reynolds.  

\begin{figure}[!ht]
    \centering
    \includegraphics[width=.8\linewidth]{figuras/outras/placeholder.png}
    \caption{FOTO DO APP SENDO USADO\cite{autor}.}
    \label{fig:placeholder}
\end{figure}

Para se averiguar a velocidade foram utilizados dois recursos: velocímetro do próprio carro e velocidade medida pelo pelos tubos de Pitot. Idealmente as duas medidas devem apresentar resultado semelhante. Para se acompanhar a velocidade medida pelos Pitots em tempo real foi utilizado o aplicativo desenvolvido.

\begin{figure}[!ht]
    \centering
    \includegraphics[width=.8\linewidth]{figuras/testes/teste_aviao_completo.JPG}
    \caption{Foto do teste com aeronave Céu Azul 2019 P1\cite{autor}.}
    \label{fig:foto_teste_aviao}
\end{figure}

\begin{figure}[!ht]
    \centering
    \includegraphics[width=.8\linewidth]{figuras/outras/placeholder.png}
    \caption{FOTO DO TESTE - 2\cite{autor}.}
    \label{fig:placeholder}
\end{figure}

\begin{figure}[!ht]
    \centering
    \includegraphics[width=.8\linewidth]{figuras/outras/placeholder.png}
    \caption{FOTO DO TESTE - 3\cite{autor}.}
    \label{fig:placeholder}
\end{figure}

Para cada bateria de testes a asa foi instalada em um ângulo de incidência diferente. Foram realizados testes para os ângulos de 0, 3, 6, 9, 12 e 15 graus. Para cada um dos ângulos, três baterias de teste foram conduzidas, buscando-se criar uma base de dados grande e diminuir a influencia dos testes nos resultados.

